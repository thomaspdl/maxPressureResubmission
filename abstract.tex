% !TEX root = ./MParticle_resubmit.tex
\begin{footnotesize}
This work describes a type of distributed feedback control algorithm that acts on a vertical queueing network where flow dynamics may greatly outpace the rate of feedback and actuation. The modeled network has a known, finite set of feasible actuations for the binary controllers located at each network node. It also has known expected demands, split ratios, and maximum service rates. Previous work proposed the application of a max pressure controller to maximize throughput on such a network without the need for centralized computation of a control policy. Here, we extend the max pressure controller to satisfy practical constraints on the frequency of switching and guarantees on proportional actuation. We fundamentally alter the formulation of max pressure to a setting where the controller may only update at a rate significantly slower than the dynamics of queue formation. Furthermore, the set of allowable controllers is extended to any convex combination of available signal phases to account for signal changes within a single signal ``cycle''. We show that this proposed extended max pressure controllers stabilize a vertical queueing network (queue lengths remain bounded in expectation) given slightly increased restrictions on admissible network demand flows.  
This work is motivated by the application of controlling traffic signals on arterial road networks. Max pressure provides an intriguing alternative to existing feedback control systems due to its distributed implementation and theoretical guarantees, but cannot be directly applied as originally formulated due to hardware and safety constraints. We ultimately apply our extension of max pressure to a simulation of an existing arterial roadway and provide comparison to the control policy that is currently deployed on this site. 
\end{footnotesize}

%
%Designing efficient control of signalized intersections on road traffic networks is becoming an increasingly urgent issue in urban areas where vehicular traffic is nearing the capacity of available infrastructure.  Many currently deployed algorithms are not capable of reacting to changes in demand beyond gradual ``time of day'' variations, and thus cannot effectively handle the surges caused by nearby incidents or other common but unpredictable events. This work provides a distributed, reactive control algorithm that acts on a vertical queueing network controlled by binary signals at each internal node. The modeled network has known expected demands, split ratios and maximum service rates. Previous work proposed the application of a max pressure controller to maximize throughput on such a network without the need for centralized computation of a control policy. Here, we extend the max pressure controller to a more practical control scenario where actuator hardware and safety regulations limit the frequency of actuation relative to the more rapid timescale of queue formation and dissipation. Specifically, we extend the feedback cycle so that control can only be modified at a rate much slower than that of the network's queue dynamics. We then extend the set of allowable controllers in this setting to any convex combination of available signal phases to account for signal changes within a single signal ``cycle''. We show that the proposed extended max pressure controllers stabilize the network (queue lengths remain bounded in expectation) given slight restrictions on admissible network demand flows.  
%
