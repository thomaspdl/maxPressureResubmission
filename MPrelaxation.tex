% !TEX root = ./MParticle_resubmit.tex
\section{A Cycle-based Max Pressure Controller} 
For safety reasons, an intersection controller cannot switch signal phase actuation immediately. Instead, it must incorporate a pause of $R\approx2.5$ seconds in which all signal phases have a red light. This \emph{clearance time} allows all vehicles in the previously actuated phase to clear the intersection before a conflicting phase can be permitted to use the intersection. In the standard formulation of max pressure, the controller chooses an appropriate action based feedback received at every time step of the modeled dynamics. To accurately capture queuing behaviors observed on arterial roadways, a model would need to operate with a time discretization of $\Delta t< 10$ seconds. A signal switch at every time step could therefore result in more than 25\% loss of intersection service capacity, which is not considered in the theoretical examination presented in \cite{MaxPressureStochastic}. 

In this work, we explicitly specify the number of signal switches that occur in a fixed number of model time steps using the familiar concept of a \emph{signal cycle}. As typical with modern traffic signals, the \emph{cycle-based max pressure controller} rotates through all available signal phases within a known time period. We define \emph{cycle time} $\tau$ as a predefined number of model time steps and require that each controller phase $S$ must be green for some proportion $\lambda_S \geq \kappa_S$ of the $\tau$ steps, where the \emph{minimum green splits} $\kappa_S \in (0,1) \; \forall \; S \in U $ are parameters selected by a network manager to enforce equity in movement actuation.  

The selection of a cycle time $\tau$ intuitively affects intersection capacity. Our proof of network stability in the following sections relies on the fact that road links are \emph{undersaturated}: that is, the expected demand is served (on average) within a signal cycle. To avoid link saturation, 
we pose the following convex optimization problem (extended from that in \cite{MaxPressureStochastic}) to determine minimum constrained feasible actuation time $\Lambda^*$:
\begin{equation} \label{sum_lambda}
\begin{aligned}
\Lambda ^{*} = & \min_{\lambda = \{\lambda_S\}}
& & \sum_{S\in U} \lambda_{S} \\
& \text{subject to}
& &  \lambda_{S} \geq \kappa_S \; \forall \; S\in U\\
&&& f_{l}r(l,m) < \sum_{S}\lambda_{S} c(l,m)S(l,m)
\end{aligned}
\end{equation}
where $\kappa_S \in [0,1] \; \forall S\in U$ and $\sum_S \kappa_S <1$. If $\Lambda^* > 1$, the demand is not feasible under the set of $\{\kappa_S\}$ for any cycle length. If $\Lambda^* < 1$, then we can define a cycle length for which flow is admissible without link saturation. However, this cycle length $\tau$ must be significantly greater than $\Lambda^*$ to account for clearance times. If we define $L = \lceil (\frac{R}{\Delta t} \cdot |U|) \rceil$ to be the total number of \emph{lost time steps} per cycle, a feasible cycle length $\tau$ must satisfy the following condition: 
\begin{equation} \label{taumin} \tau > \tfrac{L}{1-\Lambda^*} \end{equation} 
%\begin{itemize}
%\item 
%%each phase $S\in  U$ is allocated at least some pre-defined minimum time $\kappa_S$: 
%$\lambda_{S} \geq \kappa_S$
%\item 
%%the flow is feasible: 
%$f_{l}r_{l,m} < \sum_{S}\lambda_{S} c(l,m)S(l,m)$
%\item 
%%all of the cycle time is allocated: 
%$\sum_{S} \lambda_{S} = 1 - \frac{L}{\tau}$
%\end{itemize}
Given an appropriate $\tau$ which satisfies \eqref{taumin}, the cycle-based max pressure controller is a relaxed control matrix that is constructed as follows:
\begin{gather} \label{uc*} 
S^{r*}(t) = u^{c*} (X(t)) = \sum_{S \in U}\lambda_{S}^{*}S \text{, \quad where} \\
  \label{distMP_LP}
\{ \lambda^*_S \} = \;  \underset{\lambda_{1},...,\lambda_{\vert U\vert}}{ \arg \max} \sum_{S \in U}\lambda_{S}\gamma(S)\big(X(\lfloor {t}/{\tau} \rfloor)\big)
\end{gather}
\begin{align*}
%\Big(\sum_{l,m} w(l,m)(t)c(l,m) S(l,m)\Big)  \\
\text{subject to} &\quad  \lambda_{S} \geq \kappa_s, \quad \sum_{S} \lambda_{S} \leq 1 - \tfrac{L}{\tau} 
\end{align*}
At time step $t=n\tau$ for integer $n$, the controller $u^{c*}$ uses feedback measurements $x(t)$ to select a relaxed control matrix $S^{r*}$ with components $\lambda_S^*$ that satisfy \eqref{distMP_LP}. This relaxed controller is then applied for the subsequent $\tau$ time steps $\{t, t+1, \ldots, t+\tau-1\}$ before the controller is updated. 

Note that this controller is modeled such that all phases in an intersection are simultaneously actuated at some proportion of their maximum flow capacity. This is not possible in practice, as many phases will have to make conflicting use of the same intersection resources. Hence individual phases $S$ will have to be actuated in series, with each having a duration corresponding to a number of ``time units'' that are equal to cycle proportions $(\lambda_S \tau\cdot  \Delta t)$. Feedback measurements will then be a measure of ``average'' cycle queue length acquired over a set of measurements spanning the previous cycle.  

Because cycle-based max pressure can be implemented such that phases occur in a predictable order, a controller running cycle-based max pressure can be synchronized with neighboring controllers to enforce a ``green-wave progression'' as is standard practice in existing traffic signal control design. 

\section{Stability of Cycle-Based Max Pressure}
Cycle-based max pressure is fundamentally different from the standard max pressure formulation of Variaya \cite{MaxPressureStochastic} in two ways: first, it only updates the controller once every signal cycle (or $\tau$ model time steps); second, it applies a relaxed phase actuation (which is some convex combination of standard phase actuations). This section will address how each of these modifications individually affects the stability of the resulting controlled networks. 

\subsection*{Properties of a $\tau$-updated controller}
Suppose that we impose that the control actuation $S^*(t)$ can only be updated every $\tau$ model time steps. A resulting \emph{$\tau$-updated control sequence} is composed of a single  control matrix repeated for $\tau$ time steps of the model dynamics:
\begin{align}  \label{CYCLE_CONTROLLER}
S(n\tau+1)  &= S(n\tau +2) = \ldots = S((n+1)\tau ) %\\ \nonumber
%&= S^*(n\tau +1) =  \arg\max_{S\in U} \{\gamma(S)(X(n\tau +1 ))\}  
\end{align}
%
%\subsection*{Relaxed Max Pressure Controller}
%
%Let $\tau \in \mathbb{N}$ be fixed, the number of time steps between two actuation of the controller
%$\forall n \in \mathbb{N}$, 
%
%\begin{align} \nonumber
%S(n\tau&+1)   = \ldots = S((n+1)\tau ) = S^*(n\tau +1) \\
%&=  \arg\max\{\gamma(S)(X(n\tau +1 )) \vert S \in U\}  \\
%& =  \underset{\lambda_{1},...,\lambda_{\vert U\vert}}{ \arg \max}  \sum_{S \in \mathcal{U}}\lambda_{S}\Big(\sum_{l,m} w(l,m)(t)c(l,m) S(l,m)\Big)  \\
%\nonumber  \text{subject to}\\
%\nonumber    \lambda_{S} & \geq \kappa_s\\
% \sum_{S} \lambda_{S} &  = S^{r*} = \displaystyle\sum_{S \in \mathcal{U}}\lambda_{S}^{*}S \leq 1 - \tfrac{L}{\tau}  \label{distMP_LP_2} 
%\end{align}
In Appendix \ref{tauadmissible}, we prove that the set of demands that can be accommodated using $\tau$-updated control sequences is the same set of feasible flows as in \eqref{feasible_demand}. This equivalence becomes intuitive when one considers that our definition of feasible flows considers only the long-term (more precisely, infinite-term) average of both demand and service rates, and any infinite control sequence with limited admissible phases can be re-arranged to form a $\tau$-updated sequence for some $\tau$. 
% note that a $\tau$ control matrix in $\tilde S$ is simply the average of the corresponding $\tau$ matrices in $\overline S$, such that $M_{\tilde S} = M_{\overline S}$. Hence,
%\begin{align}
%f_{l} r(l,m) < c(l,m)M_{\overline{S} }(l,m) \implies \\ f_{l} r(l,m) < c(l,m)M_{\tilde{S} }(l,m). \nonumber
%\end{align}
As we will show in the following sections, the only additional impact of occasional updating will be an increased bound on queue lengths relative to the standard max pressure setting.
%, which immediately seems counterintuitive. However, note that our current model framework assumes infinite link buffer capacities. With the $\tau$-non updated controller, instantaneous link queues will be higher but the same \emph{long-term average} service rates can be achieved -- and thus by our definition, this flow can be equally accommodated by either controller type. 
%We have shown that the same flows are ``admissible'' when control sequences updated only every $\tau$ time steps as when they are updated every time step. 
%During a single time step, the average flow arriving at queue $(l,m)$ is $f_{l}r(l,m)$ and the maximum number of vehicles capable of leaving is $c(l,m) M(l,m)$. Similarly, the average flow arriving between time steps $t$ and $t+\tau$ is $\tau f_{l} r(l,m) $, and the number of vehicles capable of leaving is $\tau c(l,m) M(l,m)$. So if a flow is admissible using an updated control sequence $\overline S$%with average service $M_{\overline S}$
%, it can be similarly accommodated with a non-updated sequence $\tilde S$, where each consecutive set of
\subsection*{Stability of cycle-based max pressure}
Here we examine the stability of the cycle-based max pressure controller using $\tau$-updated sequences of relaxed controllers with fixed minimum phase proportion constraints, as formulated in \eqref{uc*}-\eqref{distMP_LP}.

Define $conv_{\kappa}$ as the set of convex combinations of control matrices with coefficients larger than $\kappa$:
\begin{equation}
conv_{\kappa} = \Big\{ \sum_{S}\lambda_{S}S \big| \; \lambda_S > \kappa_S \; \forall S\in U\Big\}
\end{equation}
Also define a set of \emph{undersaturated} admissible demands $D_{\kappa}$ with elements $d$ such that $f=dP$ and 
\begin{equation}\label{admissible_relaxed}
 f_{l}r(l,m) < c(l,m)S^r(l,m) 
 \end{equation} 
This condition (also seen in \eqref{sum_lambda}) ensures that a demand $d\in D_{\kappa}$ can in average be served \emph{within a single cycle} by a relaxed control matrix that maintains a specified minimum time allocation for each phase. 
%\begin{align} \nonumber
%d \in D_{\kappa} \; & \text{iff} \; \exists \; S^r \in conv_{\kappa} \; \\ & \text{such that} \; f_{l}r(l,m) < c(l,m)S^r(l,m)
%\label{admissible_relaxed}
%\end{align} 
\begin{Thm}\label{stabRelaxMP}
The cycle-based max pressure controller defined in \eqref{uc*}-\eqref{distMP_LP} 
%using $\tau$-updated sequences of relaxed controllers 
stabilizes a network whenever the demand is within a set of feasible undersaturated demands $D^{\kappa}$.
\end{Thm}
The remainder of this section proves Theorem \ref{stabRelaxMP} by finding a bound on \eqref{stability_condition} given a cycle-based max pressure controller. Begin by considering the expectation of the following function of queue state perturbation conditioned on the past queue state:
\begin{align} \label{alpha_beta_defs}
\vert X(t+1) \vert^2 & - \vert X(t)\vert^2 = \vert X(t) + \delta (t) \vert^2 - \vert X(t)\vert^2 \\
\nonumber &= 2X(t)^{T}\delta(t) + \vert \delta(t)  \vert^2 = 2\alpha(t) + \beta(t) 
\end{align}
with $\delta(t) = X(t+1) - X(t)$, $\alpha(t)  = X(t)^{T}\delta(t)$, and  $\beta(t) = \vert \delta (t) \vert^{2}$.
We continue by addressing bounds on $\beta$ and $\alpha$ separately. 

First we consider $\beta(t) = \vert \delta (t)\vert^{2}$.

%\subsection*{Bound on $\beta(t) = \vert \delta (t)\vert^{2}$}
\begin{Lem} \label{betalemma}
\begin{equation}\label{betabound}
\beta(t)  = \big| \delta(t) \big|^2 \leq NB^2 \end{equation}
where $B = \left\{ \overline{C}(l,m), \sum_{k} \overline{C}(k,l),  \overline{d}(l,m) \right\}$, 
$N$ is the number of queues in the network, $\overline{C} (l,m)$ is the maximum value of the random service rate $C(l,m)(t)$, and $ \overline{d}(l,m)$ is the maximum value of random demand $d(l,m)$.
\end{Lem}
The proof of Lemma \ref{betalemma} is exactly the same as presented in \cite{MaxPressureStochastic} and will therefore not be repeated here. Note that because these bounds hold for any arbitrary $S(l,m)(t) \in [0,1]$, this bound on $\beta$ is trivially extended to any convex combination of control matrices; hence it is still valid in our extension.
% {\color{red}(as will be shown later in this article, WHERE IS THIS SHOWN?)}.

%\subsection*{Bound on $\alpha (t) = X(t)^{T}\delta(t) $}
Now we examine a bound on $\alpha (t) = X(t)^{T}\delta(t) $. Again following \cite{MaxPressureStochastic}, we define additional sub-terms:
\begin{small}
\begin{align} \label{expected_short}
 & \mathbb{E}\{\alpha (t) | X(t)\} = \sum_{l \in \linkset,m} w(l,m)(t) \; \cdot   \\
 \nonumber  & \bigg[ f_l r(l,m) - \expectation{ \big[C(l,m)(t+1)S(l,m)(t)\wedge x(l,m)(t)\big] \big| X(t)}  \bigg]   \\  
 & \qquad \qquad = \alpha_1 (t) + \alpha_2 (t) \text{, \quad with} \nonumber
\end{align}
\begin{align} \label{a1_def}
\alpha_1 (t) &=  \sum_{l \in \linkset,m} \left[ f_l r(l,m) - c(l,m)S(l,m)(t) \right] w(l,m)(t) \\
 \alpha_2 (t) & =  \sum_{l \in \linkset,m}S(l,m)(t)w(l,m)(t)  \; \cdot   \\ 
&\Big[ c(l,m)- \expectation{ \big[C  (l,m)(t+1)\wedge x(l,m)(t)\big] \big| X(t) }  \Big]  
\end{align}
\end{small}
%These $\alpha_{\{1,2\}}$ are derived by first including in \eqref{expected_short} the $0$-valued term $[c(l,m)S(l,m)(t)w(l,m)(t) - c(l,m)S(l,m)(t)w(l,m)(t)]$, and then assuming that $S(l,m)(t) \in \{0,1\}$ (hence it can be brought outside of the internal min function in $\alpha_2$ without changing the ultimate result). %The later assumption does not hold in our modified controllers. 
\begin{Lem} \label{alpha2bound}
For all $l$, $m$, $t$, 
\begin{equation} \label{a2eqn}
\alpha_2 (t) \leq \sum_{l \in \linkset,m} c(l,m)\overline{C} (l,m)
\end{equation}
\end{Lem}
%\underline{Proof of Lemma \ref{alpha2bound}:} \\
%By Jensen's inequality, 
%\begin{align*}
%\expectation{C(l,m)(t+1) & \wedge x(l,m)(t) \big| X(t)} \\
%& \leq \expectation{C(l,m)(t+1) \big| X(t)}  \wedge x(l,m)(t) \\
%&= c(l,m)\wedge x(l,m)(t) \\ 
%&\leq c(l,m)
%\end{align*}
% Furthermore, we know that the term 
%$ \left[ c(l,m)- \expectation{ \big[C(l,m)(t+1)\wedge x(l,m)(t)\big] \big| X(t) }  \right] $
%is non- negative, and only equal to $0$ when $x(l,m)(t)>\overline C (l,m)$. 
%Using these relations and the observations that $w(l,m)(t) \leq x(l,m)(t)$ and $S(l,m)(t)\in [0,1]$, the following must hold 
%\begin{align*}
%\alpha_2 (t) & =  \sum_{l \in \linkset,m}  S(l,m)(t)w(l,m)(t) \; \cdot  \\
% & \qquad  \left[ c(l,m)- \expectation{ \big[C(l,m)(t+1)\wedge x(l,m)(t)\big] \big| X(t) }  \right] \\
%& \leq \sum_{l \in \linkset,m} S(l,m)(t)x(l,m)(t) \; \cdot \\ 
%& \qquad \left[ c(l,m)- \expectation{ \big[C(l,m)(t+1)\wedge x(l,m)(t)\big] \big| X(t) }  \right] \\
%& \leq \sum_{l \in \linkset,m} c(l,m) \overline{C}(l,m)
% \end{align*}
The proof of Lemma \ref{alpha2bound} again directly follows that presented in \cite{MaxPressureStochastic}; an extension from a binary controller $S\in \{0,1\}$ to a relaxed controller $S^r\in [0,1]$ is trivial. 

In fact, the extension made here only affects the $\alpha_1 (t)$ term. To demonstrate a bound on $\alpha_1(t)$ given application of a cycle-based max pressure controller $u^{c*}$, we first examine the stability of a standard max pressure controller using relaxed controllers with minimum phase proportion constraints and a stricter limitation on network demands. We will then show that a $\tau$-updated cycle-based max pressure controller also stabilizes a network, but results in a larger bound on queue lengths. 
%%%%%%%%%%%%%%%%%%%%%%%%%%%%%%%%%%%%%%%%%%%%%%%%%%%%%%%%%%%%%%%%%
%%%%%%%%% %%%%%%%%%%%%%     BEGIN REAL PROOF HERE    %%%%%%%%%%%%%%%%%%%%%%%%%
%%%%%%%%%%%%%%%%%%%%%%%%%%%%%%%%%%%%%%%%%%%%%%%%%%%%%%%%%%%%%%%%%

% 1) relaxed. 
Define an intermediate ``relaxed max pressure'' policy in which relaxed controllers are applied at the standard max pressure update rate (once per time step of the model dynamics). 
This situation was suggested in \cite{MaxPressureStochastic} to simulate enforcing minimum phase proportions in a cycle formulation of max pressure. Yet it still unrealistically models ``cycle'' updates at the same rate as the model of queueing and discharging behaviors on a realistic traffic network (hence the introduction of the $\tau$-updated formulation in this work). Nonetheless, we use this intermediate formulation to demonstrate that queue stability is still achieved upon use of a relaxed controller. 

 \begin{Lem} \label{alpha1bound}
If a ``relaxed'' max pressure control policy $S^{r*}$ is updated and applied at each time step $t$ and the demand $d$ is in the set of feasible undersaturated demands $D^\kappa$, then there exists an $\varepsilon>0$, $\eta>0$ such that 
\begin{equation} \label{a1eqn} 
\alpha_1(t)  \leq -\varepsilon \eta \big| X(t)\big| 
\end{equation}
\end{Lem}
\begin{IEEEproof}
Consider the relaxed max pressure control matrix $S^{r*}$ defined in \eqref{uc*} for $\tau =1$. By construction, $ \forall \; S^r \in conv_{\kappa}$
\begin{align} \nonumber
\sum_{l,m}c(l,m)& w(l,m)(X(t))S^r (l,m) \\
&\leq \sum_{l,m}c(l,m)w(l,m)(X(t))S^{r*}(l,m) 
\end{align}
with equality only if $S^r = S^{r*}$. 
Thus $\forall \; (S^r  \in conv_{\kappa}) \neq S^{r*}$,
\begin{align} \nonumber
 \sum_{l,m}&\big[f_{l}r(l,m) - c(l,m)S^{r*}(l,m)(t)\big]w(l,m)(X(t)) \\
&<   \sum_{l,m}\big[f_{l}r(l,m) - c(l,m)S^r(l,m)\big]w(l,m)(X(t))
\end{align}
If the demand flow is admissible according to \eqref{admissible_relaxed}, then 
$\exists \; \hat{S} \in conv_{\kappa}$ and some small $\varepsilon>0$ such that 
\begin{equation} \nonumber
c(l,m)\hat{S} (l,m) = \begin{cases}
        f_{l}r(l,m) + \varepsilon & \text{ if } w(l,m)(X(t)) > 0 \\
        0 & \text{ otherwise}
    \end{cases}
\end{equation}
Therefore, 
\begin{align} \nonumber
 \alpha_1(t) &= \sum_{l,m}\big[f_{l}r(l,m) -  c(l,m)S^{r*}(l,m)(t)\big] w(l,m)(X(t))  \\ \nonumber
 &< \sum_{l,m}\big[f_{l}r(l,m) -  c(l,m)\hat{S}(l,m)(t)\big] w(l,m)(X(t)) \\ 
&  = -\varepsilon \sum_{l \in \linkset,m} \max\{w(l,m)(X(t)) ,0\} \\ \nonumber
&\qquad \qquad +  \sum_{l \in \linkset,m}  f_l r(l,m)  \min\{ w(l,m)(X(t)) ,0\}
\end{align}
We assume that by our choice of $\hat{S}$, $f_{l}r(l,m) > \varepsilon$. Hence
%\begin{equation} 
$\alpha_1(t) < -\varepsilon \sum_{l,m}w(l,m)(t)$. 
%\end{equation}
Given the linearity of \eqref{linkweight} and the known properties of $r(l,m)(t)$, it can be show that $\sum_{l,m}w(l,m)(t) \geq \eta |X(t)|$ for some $\eta >0$. This completes the derivation of \eqref{a1eqn}.
\end{IEEEproof}
For ease of notation, now combine \eqref{betabound}, \eqref{a2eqn} and \eqref{a1eqn} to obtain the following expression given application of the  ``relaxed max pressure'' controller:
\begin{align}
\expectation{&|X(t+1)|^{2} - |X(t)|^{2}  |   X(t)} = \expectation{2\alpha(t) + \beta(t)} \nonumber \\
&<  -2\varepsilon \eta \left| X(t ) \right| + 2 \sum_{l \in \linkset,m} [c(l,m)\overline{C} (l,m) ]+ NB^{2}
\end{align}
where $N$ and $B$ are as in \eqref{betabound}. Next we establish a bound on queue growth in a single time step between controller updates. 
\begin{Lem} \label{lemma_p} 
%For a given cycle (i.e. when the controller is not updated) consisting of time steps $\{t, t+1, \ldots,  t+\tau\}$,  $\forall p \in [0, \tau - 1 ]$, 
Assuming a cycle-based max pressure controller with an cycle steps $\tau$ beginning at time $t$, the following bound on state perturbation holds for all 
%\emph{steps since update} 
$p \in [0, \tau - 1]$:
%\begin{small}
\begin{gather} \nonumber
\expectation{\vert X(t + p + 1)\vert^2 - \vert X(t + p)\vert^2 \big| X(t) \ldots X(t + p )} \\ 
<  Y + h(p) - 2\varepsilon \eta \vert X(t + p )\vert   \label{boundCycleMP} \\ %\label{pexpect} \\ %\label{boundCycleMP} 
%\nonumber
%& = 2 \sum_{l,m} c(l,m)\overline{C}(l,m) + N B^2 -2\varepsilon \eta \vert X(t + p )\vert  \\
%& \qquad \quad+ 2pNB\Big(\varepsilon\eta +  \sum_{l,m}\big[f_{l}r(l,m) + c(l,m)\big]\Big)  \label{boundCycleMP} \\
%\end{align}\begin{gather}
\text{ \normalsize{ for }\quad} Y = 2 \sum_{l,m} c(l,m)\overline{C}(l,m) + N B^2  \text{ \; \normalsize{and}} \\
h(p) =  2 pNB\Big(\varepsilon\eta +  \sum_{l,m}\big[f_{l}r(l,m) + c(l,m)\big]\Big)  
\end{gather}\end{Lem}
\begin{IEEEproof}
As in Lemmas \ref{betalemma}-\ref{alpha1bound} above, begin by dividing the argument of \eqref{boundCycleMP} into three parts: $|X(t+ p + 1)|^{2} - |X(t + p)|^{2}  = 2(\alpha_1(t + p)+\alpha_2(t + p)) + \beta(t + p)$, 
%\begin{align} \label{immediate_to_bound} 
% |X(t+ p + 1)|^{2} &- |X(t + p)|^{2} \\  \nonumber
% & = 2(\alpha_1(t + p)+\alpha_2(t + p)) + \beta(t + p)   
%\end{align}
where $\beta$, $\alpha_{1}$ and $\alpha_{2}$ are quantities that depend on the controller applied at $(t + p)$:
\begin{small}
\begin{align}
\beta (t+p)  &= \vert X(t + p + 1) - X(t + p) \vert^{2} \label{pbeta} \\
\alpha_{1} (t+p) &= w(l,m)(X(t + p)) \; \cdot \label{pa1} \\  \nonumber
& \qquad \qquad  \sum_{l,m} \Big(f_{l}r(l,m) - c(l,m)S(l,m)(t)\Big)\\ 
 \alpha_{2}  (t+p)&= w(l,m)(X(t + p)) \cdot \sum_{l,m} \bigg(c(l,m)S(l,m)(t) \label{pa2} \\ \nonumber
 - \expectation{ & \big[C(l,m)(t+p + 1)\wedge x(l,m)(t + p)\big] \big| X(t + p) }\bigg)
 \end{align}
 \end{small}Bounds on the expectations of  $\beta(\cdot)$ and $\alpha_2(\cdot)$ were previously established for any binary or relaxed control matrix in \eqref{betabound} and \eqref{a2eqn}, respectively. Thus we already know that:
\begin{align}\nonumber
\expectation{&\vert X(t + p + 1) \vert^2 - \vert X(t + p)\vert^2 \vert X(t) \ldots X(t + p - 1)} \\  \label{cbmpcombined1}
&< 2 \sum_{l,m} c(l,m)\overline{C}(l,m) + N B^2 + \expectation{2\alpha_1(t+p) }
\end{align}
%\begin{align}
%%\label{a1bound_original} \alpha_1 &<  -\varepsilon \eta \left| X(t ) \right| \\ 
%\label{a2bound_original} \alpha_2 (\cdot)&<  \sum_{l \in \linkset,m} c(l,m)\overline{C} (l,m) \\ 
%\label{bbound_original} \beta (\cdot)  &<  N \displaystyle\sum_{l,m} \max\left\{ \overline{C}(l,m), \sum_{k} \overline{C}(k,l),  \overline d (l,m) \right\}^2
%\end{align}
The remainder of the bound proposed in \eqref{boundCycleMP} originates from the $2 \alpha_1 (t+p)$ term, which is directly dependent on the explicit form of the controller $S$. Rewrite the expression for $2\cdot \alpha_1$ in \eqref{pa1} as follows:
%\begin{align}
%\nonumber
%&\expectation{\vert X(t + p + 1)\vert^2 - \vert X(t + p)\vert^2 \vert X(t) \ldots X(t + p)} \\
%\nonumber
%& = \expectation{ 2\alpha_{1} (t+p) + 2\alpha_{2}(t+p) +\beta (t+p)    \vert  X(t) \ldots X(t + p)  }\\
%\nonumber
%&< \expectation{2\alpha_{1} (t+p) \vert X(t) \ldots X(t + p) } + K \\
%&= 2\sum_{l,m} {[f_{l}r(l,m) - c(l,m)S(l,m)(t) ]w(l,m)(X(t + p))} + K \label{withK}
%\end{align}
%%Therefore we try to show that  $\forall p \in [0,\tau - 1 ]$, $\exists C(p)$ such that
%\begin{equation}
%\expectation{\alpha_{1}(t + p)\vert X(t)} < -\varepsilon \vert X(t) \vert  + C(p)
%\end{equation}
%$2\sum_{l,m} [f_{l}r(l,m) - c(l,m)S(l,m)(t) ]w(l,m)(X(t + p) )$:
\begin{small}
\begin{align} \nonumber 
&2\sum_{l,m} w(l,m)(X(t + p)) [f_{l}  r(l,m) -  c(l,m)S(l,m)(t) ]     \\ \nonumber
&=  2\sum_{l,m} w(l,m)(X(t )) [f_{l}r(l,m) - c(l,m)S(l,m)(t) ]  \\ \nonumber
& \qquad  \qquad + 2\sum_{l,m} \Big\{w(l,m) \Big( X(t + p) - X(t) \Big)\; \cdot  \\ 
& \qquad \qquad \qquad \qquad \quad [f_{l}r(l,m) - c(l,m)S(l,m)(t) ] \Big\} \\  
&=  \xi_{1} (t,p,S)+  \xi_{2} (t,p,S) \text{\quad for} \nonumber
%\label{2terms}
\end{align}
\begin{align*} 
\xi_{1} (t,S) = &2\sum_{l,m} w(l,m)(X(t)) [f_{l}r(l,m) - c(l,m)S(l,m)(t) ]  \\
\xi_{2} (t,p,S) &= 2\sum_{l,m} \Big\{w(l,m) \Big( X(t + p) - X(t) \Big) \cdot  \\ 
& \qquad \qquad \qquad \qquad [f_{l}r(l,m) - c(l,m)S(l,m)(t) ] \Big\} 
\end{align*}
\end{small}By Lemma \ref{alpha1bound} we know that 
$
\xi_1(t,S) < -2\varepsilon \eta \vert X(t) \vert
$. Because $
\vert X(t) \vert = \vert X(t + p) - (X(t + p) - X(t)) \vert
%&> \Big| | X(t + p) \vert - \vert  X(t + p) - X(t)   |  \Big| \\
>  \vert X(t + p) \vert - \vert  X(t + p) - X(t)   \vert$, 
we find that
\begin{align} \nonumber
\xi_1 &(t,p,S) < -2\varepsilon \eta \big( \vert X(t + p) \vert -  \vert  X(t + p) - X(t)   \vert \big)\\ \nonumber
&< -2\varepsilon \eta   \vert X(t + p) \vert  + 2\varepsilon \eta\sum_{i=1}^{p} \vert  X(t + i) - X(t + i - 1)   \vert  \\ 
&= -2\varepsilon \eta   \vert X(t + p) \vert + 2\varepsilon \eta\sum_{i=1}^{p} \vert \delta(t + i -1) \vert 
\label{sump}
\end{align}
So by \eqref{sump} and \eqref{betabound}, 
\begin{align} \nonumber
\xi_1(t, S) &< -2\varepsilon \eta  \vert X(t + p) \vert \\ \nonumber
& \qquad + 2\varepsilon \eta  p \sum_{l,m} \max\left\{ \overline{C}(l,m),\sum_{k}\overline{C}(k,l),\overline d(l,m) \right\} \\ 
& = 2 \varepsilon \eta \cdot  \Big(pNB -  \vert X(t + p) \vert\Big) \label{xi1_bound}
\end{align}
%%%%Plugging \eqref{first_term} into \eqref{withK}, we have
%%%%\begin{align} \nonumber
%%%%&\expectation{\vert X(t + p +  1)\vert^2 - \vert  X(t + p)\vert^2 \vert X(t),\ldots,X(t + p)}  \\ \nonumber
%%%%& < K  -2 \varepsilon \eta \vert X(t + p) \vert  \\ \nonumber
%%%%&\qquad + 2\varepsilon \eta p \sum_{l,m} \max\left\{ \overline{C}(l,m),\sum_{k}\overline{C}(k,l),\overline d(l,m) \right\} \\ \nonumber
%%%%&  \qquad +2 \sum_{l,m} [f_{l}r(l,m) - c(l,m)S(l,m)(t) ]\\
%%%%&\qquad \qquad \cdot\Big(w(l,m)(X(t + p) ) - w(l,m)(X(t))\Big) \label{combined1}
%%%%\end{align}
%
%\begin{align} \nonumber
%\sum_{l,m} [f_{l}  r(l,m) - & c(l,m)S(l,m)(t) ]  w(l,m)(t + p)   \\ \nonumber
%& =   \sum_{l,m} [f_{l}r(l,m) - c(l,m)S(l,m)(t) ]w(l,m)(t )  \\
%&\qquad \qquad \qquad +\sum_{l,m} [f_{l}r(l,m) - c(l,m)S(l,m)(t) ](w(l,m)(t + p ) - w(l,m)(t)) 
%\label{2terms}
%\end{align}
%%%%\underline{Bound on $\xi_{2}$} \\
%We now have to bound the term
%\begin{align}\nonumber
%\xi_2(t,p,S) &= 2\sum_{l,m} [f_{l}r(l,m) - c(l,m)S(l,m)(t) ]\\
%&\quad \cdot \Big(w(l,m)X((t + p) ) - w(l,m)(X(t))\Big)
%\label{2xi2}
%\end{align}
%
%\begin{comment}
%
%\begin{align*}
%\expectation{ \vert X(t+p)\vert^2 - \vert X(t+ p - 1)\vert^2  \vert  X(t) } &= \sum_{l,m}[f_{l}r(l,m) - c(l,m)S(l,m)]w(l,m)(X(t + p)) + \frac{K}{p^2}\\
%&=   \sum_{l,m}[f_{l}r(l,m) - c(l,m)S(l,m)]  w(l,m)(X(t + k)) + \frac{K}{p^2}\\
%%&= \tau K + \sum_{l,m}[f_{l}r(l,m) - c(l,m)S(l,m)]\sum_{k=1}^{\tau - 1}  
%%\{ w(l,m)(X(t) ) + w(l,m)(X(t + k )) - w(l,m)(X(t) ) \} \\
%&=  K + \sum_{l,m}[f_{l}r(l,m) - c(l,m)S(l,m)]w(l,m)(X(t)) \\
%&+ \sum_{l,m}[f_{l}r(l,m) - c(l,m)S(l,m)]  
%\{ w(l,m)(X(t + k) ) - w(l,m)(X(t) ) \} \\
%&< K - \varepsilon \vert X(t) \vert + \sum_{l,m}[f_{l}r(l,m) - c(l,m)S(l,m)]  
%\{ w(l,m)(X(t + k) ) - w(l,m)(X(t) ) \} \\
%\end{align*}
%
%\end{comment}
To bound $\xi_2$, we study the term
\begin{align} \nonumber
&w(l,m)(X(t+p))   -  w(l,m)(X(t)) \\ \nonumber
& = \sum_{n=1}^{p} w(l,m)(X(t+n)) - w(l,m)(X(t + n - 1)) \\ \nonumber
&= \sum_{n=1}^{p} \Big\{ x(l,m)(t + n) - x(l,m)(t + n - 1)   \\ \nonumber 
&  \quad - \sum_{s \in Out(m)}[ x(m,s)(t + n) - x(m,s)(t + n - 1) ]r(m,s)  \Big\} \\ 
&= \sum_{n=1}^{p}  w(l,m)( \delta(t + n - 1)) 
\end{align}
By \eqref{betabound} and the fact that $w(\cdot)$ is linear,  
%\begin{align} \nonumber
%\vert w(l,m)(\delta&(t + n -1)) \vert \\  \label{wdelta}
%&< \sum_{u,v} \max\Big\{  \overline{C}(u,v), \sum_{k}\overline{C}(k,u), \overline{d}(u,v)\Big\}
%\end{align}
\begin{equation} \label{wdelta}
\vert w(l,m)(\delta(t + n -1)) \vert <NB
\end{equation}
Plugging \eqref{wdelta} back into the definition of $\xi_2$, we obtain
\begin{small}
\begin{align} \nonumber 
\xi_2&(t,p,S) = 2\bigg( \sum_{l,m}\big([f_{l}r(l,m) - c(l,m)S(l,m)] \; \cdot \\  \nonumber
&\qquad \qquad \qquad  \sum_{n=1}^{p}  w(l,m)( \delta(t + n - 1))\bigg)  \\ \nonumber
&< 2 \sum_{n=1}^{p}  \sum_{l,m}[f_{l}r(l,m) - c(l,m)S(l,m)] \; \cdot \\ \nonumber
&\qquad \qquad \sum_{u,v} \max\Big\{  \overline{C}(u,v), \sum_{k}\overline{C}(k,u), \overline{d}(u,v)\Big\} \\
& = 2NB \sum_{n=1}^{p}  \sum_{l,m}[f_{l}r(l,m) - c(l,m)S(l,m)] \label{2xi2_bound1}
\end{align}
\end{small}Also note that 
\begin{equation*}
\Big| \sum_{n=1}^{p}  \sum_{l,m}[f_{l}r(l,m) - c(l,m)S(l,m)] \Big| < p \sum_{l,m}[f_{l}r(l,m) + c(l,m)]
\end{equation*}
so \eqref{2xi2_bound1} becomes
\begin{equation} \label{xi2_bound}
\xi_2(t,p,S) < 2 NB p \cdot \Big( \sum_{l,m}[f_{l}r(l,m) + c(l,m)]\Big)
%\\ \nonumber &\cdot\left( \sum_{l,m} \max\Big\{  \overline{C}(l,m), \sum_{k}\overline{C}(k,l), \overline{d}(l,m)\Big\} \right)
\end{equation}
Substituting \eqref{xi1_bound} and \eqref{xi2_bound} into \eqref{cbmpcombined1} yields \eqref{boundCycleMP}.
\end{IEEEproof}\vspace{-1em}

Given Lemma \ref{lemma_p}, we show that for a time $t$ within any number $K$ of $\tau$-updated cycles, the following quantity is bounded:
\begin{align} \nonumber 
&\sum_{t = 1}^{K \tau} \expectation{\vert X(t + 1)\vert^2 - \vert X(t)\vert^2 \vert X(t)}\\ \nonumber
&=\sum_{t=1}^{K-1} \sum_{p=0 }^{\tau - 1} \expectation{\vert X(t + p + 1)\vert^2 - \vert X(t + p)\vert^2 \vert X(t + p)}\\ \nonumber
&< \sum_{t=1}^{K-1} \sum_{p=0 }^{\tau - 1} (Y+ h(p) -2\varepsilon\eta \vert X(t + p )\vert ) \\
&< -2\varepsilon \eta \sum_{t=1}^{K \tau} \vert X(t )\vert  + (K - 1) \Big(\tau Y+\sum_{p=0 }^{\tau - 1} h(p) \Big) \label{equivalencestabcycleMP}
\end{align}
which, when taking the expectation, yields
\begin{align} \nonumber 
& \expectation{\vert X(K \tau + 1)\vert^2 - \vert X(1)\vert^2 } \\
& <  -2\varepsilon\eta \sum_{t=1}^{K \tau} \expectation{\vert X(t )\vert}+ (K - 1) ( \tau Y+\sum_{p=0 }^{\tau - 1} h(p) )
\end{align}
Rearranging gives
\begin{small}
\begin{align} \nonumber 
\dfrac{1}{K \tau}& \sum_{t=1}^{K \tau}  \expectation{\vert X(t )\vert} < \frac{1}{2\varepsilon\eta K \tau} \expectation{\vert X(1)\vert^2 - \vert X(K \tau + 1)\vert^2  } \\
&  \qquad \qquad \qquad \qquad \quad + \frac{\tau-1}{2\varepsilon\eta K\tau }   \Big(\sum_{p=0 }^{\tau - 1} h(p) + \tau Y \Big) \Big] \\
&< \dfrac{1}{2\varepsilon\eta\ K \tau}\expectation{\vert X( 1)\vert^2} + \dfrac{ 1}{2\varepsilon\eta \tau} \left(\sum_{p=0 }^{\tau - 1} h(p) + \tau Y\right) \label{finalbound}
\end{align}
\end{small}
By \eqref{stability_condition}, the bound 
\begin{small}
\begin{equation*} \label{CYCLE_BOUNDS} 
%\dfrac{1}{K \tau} \sum_{t=1}^{K \tau}  &\expectation{\vert X(t )\vert} < \\ \nonumber
%& \frac{1}{2\varepsilon\eta\ K\tau }\expectation{\vert X( 1)\vert^2} + \frac{ 1}{2\varepsilon\eta \tau} \left(\sum_{p=0 }^{\tau - 1} h(p) + \tau Y\right) 
\dfrac{1}{K \tau} \sum_{t=1}^{K \tau}  \expectation{\vert X(t )\vert} <  \frac{1}{2\varepsilon\eta \tau }\bigg[ \tfrac{1}{K} \expectation{\vert X( 1)\vert^2} +  \sum_{p=0 }^{\tau - 1} h(p) + \tau Y  \bigg]
\end{equation*}
\end{small}establishes that the cycle-based max pressure controller $u^{c*}(X(t))$ defined in \eqref{uc*} will stabilize a vertical queueing network with dynamics $X(t)$ as in \eqref{entrydynamics}-\eqref{internaldynamics}. However, the maximum expected queue lengths will be higher in the case of a cycle-based max pressure than in the standard frequent update setting. This relative increase can be shown to be linear in cycle length $\tau$. 
% {\color{red} Note that \eqref{CYCLE_BOUNDS} is larger than \eqref{IF_BOUNDS} by a factor of} 
%\begin{equation}
%\dfrac{1}{\varepsilon \tau} \sum_{p=1}^{\tau - 1} h(p) = \dfrac{\tau - 1}{2\varepsilon} \cdot\textit{[constant]}
%\end{equation}
%Hence the relative increase to queue bounds is linear in $\tau$. 


%%%%%%%%%%%%%%%%%%%%%%%%%%%%
%%%%%%%%%%%%%%%%%%%%%%%%%%%%










