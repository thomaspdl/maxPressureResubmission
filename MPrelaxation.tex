% !TEX root = ./MParticle_resubmit.tex
\section{A Cycle-based Max Pressure Controller} 




%%%%%%%%%
%relaxation of the controller
%%%%%%%%%
 

%%%%%%%%%%%
%Physically, one could consider one model time step to represent an entire light cycle, during which each signal phase will be sequentially actuated for some proportional amount of the period. (Although this would suggest a serial division of the time period between independent phases rather than the parallel implementation that is represented here mathematically.) 
%%%%%%%%%%%


The attractive qualities of max pressure for controlling vehicular traffic in urban road networks are presented in \cite{MaxPressureStochastic}. Yet as previously discussed, this original formulation of the controller is not practical for application on a signalized traffic network for three reasons:
\begin{enumerate}[a)]
\item it does not account for capacity reductions (lost time) due to excessive signal switching, 
\item it cannot enforce coordination between subsequent intersections for purposes of maximizing flow continuity, and
\item it does not provide guarantees that low-demand queues will be served within a finite time period. 
\end{enumerate} 
These limitations motivate our extension of the standard immediate feedback max pressure control algorithm. In the following section, we define a new \emph{cycle-based} max pressure controller which bounds the number of signal switches per fixed time period, provides capacity for standard signal coordination methods, and can easily guarantee a minimum service rate for all intersection phases. We then show that the application of this controller yields a similar stability guarantee to that shown by Varaiya for the standard controller given slightly weaker conditions on demand flow. The structure of this proof is as follows: 
\begin{enumerate}[i.]
\item First, we formalize a calculation of the \emph{lost time} incurred by signal switching actions. 
\item Then we introduce a formulation of the cycle-based max pressure algorithm and briefly describe how it inherently rectifies issues a), b), and c) above. 
\item Next, we introduce the concept of a \emph{$\tau$-updated} controller that can only be updated once every $\tau$ model time steps (or once per \emph{cycle}), and we show that this does not impact the feasibility of demands.  
\item We finally show that queue stability holds with a cycle-based max pressure controller consisting of \emph{$\tau$-updated sequences of relaxed control matrices with minimum proportion constraints}. However, network stability requires \emph{slightly stricter conditions on demand flows} than assumed for stability using the original max pressure formulation.  
\end{enumerate}

%
%To fulfill the first condition, we allow the control to be a convex combination of the different possible control matrices.
%For the second condition we consider a control that is not updated every time steps.
%The final control that we choose is a combination of the two.



%%%%%%%%%%%%%%%%%%%%%%%%%%%
%distributed max pressure
%%%%%%%%%%%%%%%%%%%%%%%%%%%
 
\subsection*{Lost time due to switching}
For safety reasons, an intersection controller cannot switch signal phase actuation immediately. Instead, it must incorporate a pause of $R\approx2.5$ seconds in which all signal phases have a red light. This \emph{clearance time} allows all vehicles in the previously actuated phase to clear the intersection before a conflicting phase can be permitted to use the intersection. 

In the standard formulation of max pressure, the controller chooses an appropriate action based feedback received at every time step of the modeled dynamics. To accurately capture queuing behaviors observed on arterial roadways, a model would need to operate with a time discretization of $\Delta t< 10$ seconds. A signal switch at every time step could therefore result in more than 25\% loss of intersection service capacity, which is not considered in the theoretical examination presented in \cite{MaxPressureStochastic}. 

In this work, we explicitly fix the number of signal switches that can occur in a fixed number of model time steps using the familiar concept of a \emph{signal cycle}. As typical with modern traffic signals, the \emph{cycle-based} max pressure controller rotates through all available signal movements within a fixed time period. We define \emph{cycle time} $T$ as a fixed number of model time steps and require that each controller phase $S$ must be green for some proportion $\lambda_S \geq \kappa_S$ of the $T$ steps, where the \emph{minimum green splits} $\kappa_S \in (0,1) \; \forall \; S \in U $ are parameters selected by a network manager to enforce equity in movement actuation.  

The selection of a cycle time $T$ intuitively affects intersection capacity. Our proof of network stability in the following sections relies on the fact that road links are \emph{undersaturated}: that is, the expected demand is served (on average) within a signal cycle. To avoid link saturation, 
we pose the following convex optimization problem (extended from that in \cite{MaxPressureStochastic}) to determine minimum constrained feasible actuation time $\Lambda^*$:
\begin{equation} \label{sum_lambda}
\begin{aligned}
\Lambda ^{*} = & \min_{\lambda = \{\lambda_S\}}
& & \sum_{S\in U} \lambda_{S} \\
& \text{subject to}
& &  \lambda_{S} \geq \kappa_S \; \forall \; S\in U\\
&&& f_{l}r(l,m) < \sum_{S}\lambda_{S} c(l,m)S(l,m)\\
%&&& \sum_{S} \lambda_{S} < 1
\end{aligned}
\end{equation}
where $\kappa_S \in [0,1] \; \forall S\in U$, and $\sum_S \kappa_S <1$. If $\Lambda^* > 1$, the demand is not feasible under the set of control constraints $\{\kappa_S\}$ for any cycle length. If $\Lambda^* < 1$, then we can define a cycle length for which flow is admissible without link saturation. However, this cycle length $T$ must be significantly greater than $\Lambda^*$ to account for clearance times. If we define $L = \text{ceil}\big(\frac{R}{\Delta t} \cdot |U|\big)$ to be total \emph{lost time steps} per cycle, a feasible cycle length $T$ must satisfy the following condition: 
\begin{equation} T > \frac{L}{1-\Lambda^*} \label{minTime} \end{equation} 


\subsection*{Cycle-based controller formulation}

%\begin{itemize}
%\item 
%%each phase $S\in  U$ is allocated at least some pre-defined minimum time $\kappa_S$: 
%$\lambda_{S} \geq \kappa_S$
%\item 
%%the flow is feasible: 
%$f_{l}r_{l,m} < \sum_{S}\lambda_{S} c(l,m)S(l,m)$
%\item 
%%all of the cycle time is allocated: 
%$\sum_{S} \lambda_{S} = 1 - \frac{L}{T}$
%\end{itemize}

The cycle-based max pressure controller is a relaxed control matrix that is constructed as follows:
\begin{equation} \label{uc*} 
S^{r*}(t) = u^{c*} (X(t)) = \sum_{S \in U}\lambda_{S}^{*}S \text{, \quad where}
\end{equation}  
\begin{equation}  \label{distMP_LP}
\{ \lambda^*_S \} = \;  \underset{\lambda_{1},...,\lambda_{\vert U\vert}}{ \arg \max} \sum_{S \in U}\lambda_{S}\gamma(S)(X(t))
\end{equation}
\begin{align*}
%\Big(\sum_{l,m} w(l,m)(t)c(l,m) S(l,m)\Big)  \\
\text{subject to} &\quad  \lambda_{S} \geq \kappa_s\\
&\quad \sum_{S} \lambda_{S} \leq 1 - \tfrac{L}{T} 
\end{align*}
At time step $t=nT$ for integer $n$, the controller $u^{c*}$ uses feedback measurements $x(t)$ to select a relaxed control matrix $S^{r*}$ with components $\lambda_S^*$ that satisfy \eqref{distMP_LP}. This relaxed controller is then applied for the subsequent $T$ time steps $\{t, t+1, \ldots, t+T-1\}$ before the controller is updated. 

Note that this controller is modeled such that all phases in an intersection are simultaneously actuated at some proportion of their maximum flow capacity. This is not possible in practice, as many phases will have to make conflicting use of the same intersection resources. Hence individual phases $S$ will have to be actuated in series, with each having a duration corresponding to a number of ``time units'' that are equal to cycle proportions $(\lambda_S T\cdot  \Delta t)$. Feedback measurements will then be a measure of ``average'' cycle queue length acquired over a set of measurements spanning the previous cycle.  

Because cycle-based max pressure can be implemented such that phases occur in a predictable order, a controller running cycle-based max pressure can be synchronized with neighboring controllers to enforce a ``green-wave progression'' designed to minimize individual vehicle stoppage and delay for a prioritized movement. This is standard practice in existing traffic signal control design. 

\section{Stability of Cycle-Based Max Pressure}
Cycle-based max pressure is fundamentally different from the standard max pressure formulation of Variaya \cite{MaxPressureStochastic} in two ways: first, it only updates the controller once every signal cycle (or $T$ model time steps); second, it applies a relaxed phase actuation (which is some convex combination of standard phase actuations). The following two sections will separately address how each of these modifications individually affects the stability of the resulting controlled networks. 

\subsection*{Properties of a $\tau$-updated controller}
Suppose that we are given the model dynamics $X(t)$ as in \eqref{entrydynamics}-\eqref{internaldynamics} and apply the standard max pressure controller \eqref{original_MP}. However, we further impose that the control actuation $S^*(t)$ can only be updated every $\tau$ model time steps. A resulting \emph{$\tau$-updated control sequence} is composed of a single (one-phase) control matrix repeated for $\tau$ time steps of the model dynamics. Hence for any integer $n$, the controller that maximizes the pressure at time step $n\tau + 1$ is continuously applied until time step $(n + 1)\tau$:
\begin{align}  \label{CYCLE_CONTROLLER}
S(n\tau+1)  &= S(n\tau +2) = \ldots = S((n+1)\tau ) \\ \nonumber
&= S^*(n\tau +1) =  \arg\max_{S\in U} \{\gamma(S)(X(n\tau +1 ))\}  
\end{align}
%
%\subsection*{Relaxed Max Pressure Controller}
%
%Let $\tau \in \mathbb{N}$ be fixed, the number of time steps between two actuation of the controller
%$\forall n \in \mathbb{N}$, 
%
%\begin{align} \nonumber
%S(n\tau&+1)   = \ldots = S((n+1)\tau ) = S^*(n\tau +1) \\
%&=  \arg\max\{\gamma(S)(X(n\tau +1 )) \vert S \in U\}  \\
%& =  \underset{\lambda_{1},...,\lambda_{\vert U\vert}}{ \arg \max}  \sum_{S \in \mathcal{U}}\lambda_{S}\Big(\sum_{l,m} w(l,m)(t)c(l,m) S(l,m)\Big)  \\
%\nonumber  \text{subject to}\\
%\nonumber    \lambda_{S} & \geq \kappa_s\\
% \sum_{S} \lambda_{S} &  = S^{r*} = \displaystyle\sum_{S \in \mathcal{U}}\lambda_{S}^{*}S \leq 1 - \tfrac{L}{T}  \label{distMP_LP_2} 
%\end{align}

We first prove that the set of demands that can be accommodated using $\tau$-updated control sequences is the same set of feasible flows as in \eqref{feasible_demand}. 
%We first prove that this set of $\tau$-admissible flows (demands that can be accommodated using $\tau$-non updated sequences) is in fact the same set of flows that is admissible under typical updated control sequences, defined in equation \eqref{feasible_demand}. 
Given the set admissible phases $U$, define the following sets:
\begin{itemize}
\item $\mathcal U$ is the set of control sequences with distinct elements $\{S(1), S(2) \ldots S(t) \ldots |  S(\cdot) \in U\} $,
\item $\mathcal U_{\tau}$ is the set of $\tau$-updated control sequences $\{S(1), S(1), \ldots ,  S(\tau + 1),S(\tau + 1), \ldots , S(n\tau + 1), \\  S(n\tau + 1), \ldots |  S(\cdot) \in U\} $, 
\end{itemize}
Also define the following sets of \emph{long-term control proportion matrices}, which are similar to the formulation in \eqref{longterm_proportion}:
\vspace{-.5em}
\begin{small}
\begin{align*} M_{\mathcal U} = \Big\{\lim \inf_{T}\dfrac{1}{T}\sum_{t=1}^{T}  S&(t)  \Big| \{S(1), S(2), \ldots, S(t), \ldots \}\in \mathcal U \Big\} \end{align*}
\vspace{-1em}
\begin{align*} M_{\mathcal U_\tau} = \Big\{\lim&\inf_{T} \dfrac{1}{T}\sum_{t=1}^{T} S(t) \\ & \Big| \{S(1), S(1), \ldots,  S(\tau+1),S(\tau+1), \ldots\}\in \mathcal U_{\tau} \Big\} \end{align*}
\end{small}
\vspace{-1em}

By Property \ref{feasible_property}, a demand $d$ is only feasible if there exists a control sequence $\overline {S}$ such that the corresponding long-term control proportion matrix $M_{\overline {S}}$ satisfies \eqref{feasible_demand}. Here we show $M_{\mathcal U} = M_{\mathcal{U_\tau}}$, and therefore any flows that are admissible given an unrestricted controller in $\mathcal U$ can also be accommodated using a $\tau$-updated controller in $\mathcal{U}_\tau$. 

Obviously, $M_{\mathcal U_\tau}  \subset M_{\mathcal U}$. To show equality, we can also demonstrate that $M_{\mathcal U}  \subset M_{\mathcal {U}_\tau} $. Suppose there exists a control sequence $\hat{S} = \{ S(1), S(2), \ldots \} \in \mathcal{U}$. 
By definition, 
\vspace{-1em}
\begin{align*}
M_{\hat{S}} &= \lim\inf_{T} \dfrac{1}{T}\sum_{t=1}^{T} S(t)\\&= \lim\inf_{T} \dfrac{1}{\tau T}\sum_{t=1}^{\tau T}  \tilde{S}(t) \; \\ 
& \qquad \text{ where } \tilde{S} = \{S(1), S(1), \ldots, S(t), S(t), \ldots \} \\
&= \lim\inf_{T} \dfrac{1}{ T}\sum_{t=1}^{T}  \tilde{S}(t) \\
& \in M_{U_\tau}  \qquad \implies M_{\mathcal U}  \subset M_{\mathcal {U}_\tau}
\end{align*}

This result implies that a $\tau$-updated control sequence can accommodate the same set of flows as a control sequence updated at every time step. The equivalence becomes intuitive when one considers that our definition of feasible flows considers only the long-term (more precisely, infinite-term) average of both demand and service rates, and any infinite control sequence with limited admissible phases can be re-arranged to form a $\tau$-updated sequence for some $\tau$. 
% note that a $\tau$ control matrix in $\tilde S$ is simply the average of the corresponding $\tau$ matrices in $\overline S$, such that $M_{\tilde S} = M_{\overline S}$. Hence,
%\begin{align}
%f_{l} r(l,m) < c(l,m)M_{\overline{S} }(l,m) \implies \\ f_{l} r(l,m) < c(l,m)M_{\tilde{S} }(l,m). \nonumber
%\end{align}

As we will show in the following sections, the only additional impact of occasional updating will be an increased bound on queue lengths relative to the standard max pressure setting.
%, which immediately seems counterintuitive. However, note that our current model framework assumes infinite link buffer capacities. With the $\tau$-non updated controller, instantaneous link queues will be higher but the same \emph{long-term average} service rates can be achieved -- and thus by our definition, this flow can be equally accommodated by either controller type. 
%We have shown that the same flows are ``admissible'' when control sequences updated only every $\tau$ time steps as when they are updated every time step. 
%During a single time step, the average flow arriving at queue $(l,m)$ is $f_{l}r(l,m)$ and the maximum number of vehicles capable of leaving is $c(l,m) M(l,m)$. Similarly, the average flow arriving between time steps $t$ and $t+\tau$ is $\tau f_{l} r(l,m) $, and the number of vehicles capable of leaving is $\tau c(l,m) M(l,m)$. So if a flow is admissible using an updated control sequence $\overline S$%with average service $M_{\overline S}$
%, it can be similarly accommodated with a non-updated sequence $\tilde S$, where each consecutive set of




\subsection*{Stability of cycle-based max pressure}
Here we examine the stability of the cycle-based max pressure controller using $T$-updated sequences of relaxed controllers with fixed minimum phase proportion constraints, as formulated in \eqref{uc*}-\eqref{distMP_LP}.

Define $conv_{\kappa}$ as the set of convex combinations of control matrices with coefficients larger than $\kappa$:
\begin{equation}
conv_{\kappa} = \Big\{ \sum_{S}\lambda_{S}S \big| \; \lambda_S > \kappa_S \; \forall S\in U\Big\}
\end{equation}
Also define a set of \emph{undersaturated} admissible demands $D_{\kappa}$ with elements $d$ such that $f=dP$ and 
\begin{equation} f_{l}r(l,m) < c(l,m)S^r(l,m) \end{equation} 
As in \eqref{sum_lambda}, this condition ensures that a demand $d\in D_{\kappa}$ can in average be served \emph{within a single cycle} by a relaxed control matrix that maintains a specified minimum time allocation for each phase. 
%\begin{align} \nonumber
%d \in D_{\kappa} \; & \text{iff} \; \exists \; S^r \in conv_{\kappa} \; \\ & \text{such that} \; f_{l}r(l,m) < c(l,m)S^r(l,m)
%\label{admissible_relaxed}
%\end{align} 
\begin{Thm}\label{stabRelaxMP}
The cycle-based max pressure controller defined in \eqref{uc*}-\eqref{distMP_LP} 
%using $T$-updated sequences of relaxed controllers 
stabilizes a network whenever the demand is within a set of feasible undersaturated demands $D^{\kappa}$.
\end{Thm}
\begin{proof}
Consider the expectation of the following function of queue state with perturbation 
\begin{equation} \label{delta_def}
\delta(t) = X(t+1) - X(t)
\end{equation}  
conditioned on the past queue state:
\begin{align}
\vert X(t+1) \vert^2  - \vert X(t)\vert^2 &= \vert X(t) + \delta (t) \vert^2 - \vert X(t)\vert^2 \\
\nonumber &= 2X(t)^{T}\delta(t) + \vert \delta(t)  \vert^2 \\
\nonumber &= 2\alpha(t) + \beta(t) 
\end{align}
\begin{equation} \label{alpha_beta_defs}
\text{with \quad} \alpha(t)  = X(t)^{T}\delta(t)  \text{   \; and \;  }\beta(t) = \vert \delta (t) \vert^{2}
\end{equation}
We continue by addressing bounds on $\beta$ and $\alpha$ separately. 
\subsection*{Bound on $\beta(t) = \vert \delta (t)\vert^{2}$}
%Define $\overline{C}$ as the maximum realized saturation flow and $\overline{d}$ as the maximum possible value of the demand vector. 
%If $l \in \entrylinks$ and $m \in Out(l)$,
%\begin{align} \nonumber
%\big| \delta(l,m)(t)  \big| &= \bigg| -[C(l,m)(t+1)S(l,m)(t) \wedge x(l,m)(t)]  \\ \nonumber
%&\qquad \qquad + D(l,m)(t+1) \bigg| \\
%& \leq \max\left\{  \overline{C}(l,m) , \overline d (l,m) \right\} \label{deltabound1}
%\end{align}
%where $D(l,m)(t)  = D(l) (t) R(l,m)(t)$ with $D(l)(t)$ defined as the realized demand on link $l$ at time $t$. 
%This is because we know that both $C(l,m)(t+1)S(l,m)(t) \wedge x(l,m)(t)$ and $D(l,m)(t+1)$ are non-negative, so the absolute value of their difference must be less than either of the two quantities individually. 
%
%Similarly, if $l \in \linkset\backslash\entrylinks$ and $m \in Out(l)$:
%\begin{align} \nonumber  
%\big | \delta(l,&m) (t)  \big |  = \\ \label{deltabound}
% & \bigg\vert-\big[C(l,m)(t+1)S(l,m)(t) \wedge x(l,m)(t)\big]  \\  \nonumber
% & + \sum_{k}\big[C(k,l)(t+1)S(k,l)(t) \wedge x(k,l)(t)\big] R(l,m)(t+1)\bigg\vert  \\
%& \leq \max\left\{  \overline{C}(l,m) , \sum_{k} \overline{C}(k,l) \right\} \label{deltabound2}
%\end{align}
If we define $B$ as the maximum of all of the quantities $\left\{ \overline{C}(l,m), \sum_{k} \overline{C}(k,l),  \overline d (l,m) \right\}$ and $N$ as the number of queues in the network, we can derive a bound for $\beta$ which depends on only $B$ and $N$: 
\begin{equation}\label{betabound}
\beta(t)  = \big| \delta(t) \big|^2 \leq NB^2 
\end{equation}
The proof of this bound is exactly the same as in \cite{MaxPressureStochastic}, and will therefore not be repeated here. However, note that because these bounds hold for any $S(l,m)(t) \in [0,1]$, this bound on $\beta$ is trivially extended to any convex combination of control matrices; hence it is still valid in our relaxed controllers {\color{red}(as will be shown later in this article)}.

\subsection*{Bound on $\alpha (t) = X(t)^{T}\delta(t) $}
%The term $\alpha$ in \eqref{alpha_beta_defs} is explicitly defined in terms of queue state $X(t)$ as follows: 
%\begin{small}
%\begin{align} \nonumber 
% \alpha (t)  &=  X(t)^{T}\left[X(t+1) - X(t)\right] \\ \nonumber
% &=  \sum_{l \in \linkset \backslash \entrylinks ,m}\sum_{k}\left[ C(k,l)(t+1)S(k,l)(t)\wedge x(k,l)(t)\right]\\ \nonumber
%&\qquad \qquad  \qquad \qquad  \qquad  \cdot R(l,m)(t+1)x(l,m)(t) \\ \nonumber
%&  \qquad \qquad -  \sum_{l \in \linkset,m}\left[ C(l,m)(t+1)S(l,m)(t)\wedge x(l,m)(t)\right] \\ \nonumber
%&\qquad \qquad-\sum_{l \in \entrylinks ,m} d(l,m)(t+1)x(l,m)(t)\\ \nonumber
%&= \sum_{l \in \linkset,m}\big[C(l,m)(t+1)S(l,m)(t)\wedge x(l,m)(t)\big] \\ \nonumber 
%& \qquad \qquad \cdot \Big(-x(l,m)(t)+  \sum_{p}R(m,p)(t+1)x(m,p)(t)\Big) \\  
% & \qquad \qquad + \sum_{l \in \entrylinks,m} d(l,m)(t+1)x(l,m)(t)  
% \label{alpha_explicit}
%\end{align}
%\end{small}
%{\color{red}Note that only the expectation of these terms appear in equation \eqref{stability_sufficient}, so we are interested in $\mathbb{E}\{\alpha(t) | X(t)\}$. } We therefore make the following observation: because $R(m,p)(t+1)$ is independent of $C(l,m)(t+1)$ and $X(t)$, 
%\begin{align*}
%\nonumber \expectation{&\big[C(l,m)(t+1)S(l,m)(t)\wedge x(l,m)(t)\big] \\
%\nonumber & \qquad \quad  \;   \cdot R(m,p)(t+1)x(m,p)(t) \big| X(t)}  \\
%\nonumber & = \expectation{\big[C(l,m)(t+1)S(l,m)(t)\wedge x(l,m)(t)\big]  \big| X(t)} \\
%\nonumber & \qquad \quad  \; \cdot r(m,p)(t+1)x(m,p)(t)
%\end{align*}
%Also, the expectation of demand $d(l,m)$ is equal to the measured demand $d_{l}$ on link $l$, times the relevant expected split ratio $r(l,m)$. 
%Hence the desired expectation of \eqref{alpha_explicit} can be expressed as
%%\begin{small}
%\begin{align*} 
%\nonumber \mathbb{E} & \{\alpha (t) | X(t)\}= \sum_{l \in \entrylinks,m} d_l r(l,m) x(l,m)(t)  \\
%\nonumber & + \sum_{l \in \linkset,m} \expectation{ \big[C(l,m)(t+1)S(l,m)(t)\wedge x(l,m)(t)\big]\big| X(t)}\\
%\nonumber & \qquad \qquad    \cdot \Big(-x(l,m)(t) + \sum_{p}r(m,p)(t+1)x(m,p)(t)\Big) 
% \end{align*}
% \begin{align}
% \label{Ea} = \sum_{l \in \entrylinks,m} &d_l r(l,m) x(l,m)(t)  - \sum_{l \in \linkset,m} w(l,m)(t) \; \cdot  \\
%\nonumber & \expectation{ \big[C(l,m)(t+1)S(l,m)(t)\wedge x(l,m)(t)\big] \big| X(t)}
%\end{align}
%%\end{small} 
%where $w(l,m)(t) = w(l,m)(X(t))$ is the weight of a link as defined in \eqref{linkweight}. 
%We also can include the following relation: 
%\begin{align*}
%\sum_{l \in \linkset,m} f_l &r(l,m) w(l,m)(t) \\ 
%& = \sum_{l \in \linkset,m} f_l r(l,m) \left[ x(l,m) - \sum_p r(m,p)x(m,p)(t) \right] \\
%& = \sum_{l \in \linkset,m} f_l r(l,m)x(l,m)(t) \\
%&\qquad \quad \; - \sum_m  \left[ \sum_{l\in\linkset} f_l r(l,m) \sum_p r(m,p)x(m,p)(t) \right] \\
%& = \sum_{l \in \linkset,m} f_l r(l,m)x(l,m)(t)\\
%&\qquad \qquad  - \sum_{m\in \linkset \backslash \entrylinks, p}   f_m r(m,p)x(m,p)(t) \\ 
%& =  \sum_{l \in \entrylinks,m} d_l r(l,m)x(l,m)(t)
%\end{align*}
Again following \eqref{MaxPressureStochastic}, we define the following sub-terms:
\begin{small}
\begin{align} \label{expected_short}
 & \mathbb{E}\{\alpha (t) | X(t)\} = \sum_{l \in \linkset,m} w(l,m)(t) \; \cdot   \\
 \nonumber  & \bigg[ f_l r(l,m) - \expectation{ \big[C(l,m)(t+1)S(l,m)(t)\wedge x(l,m)(t)\big] \big| X(t)}  \bigg]   \\  
 & \qquad \qquad = \alpha_1 (t) + \alpha_2 (t) \nonumber
\end{align}
\end{small}
with 
\begin{small}
\begin{align} \label{a1_def}
\alpha_1 (t) &=  \sum_{l \in \linkset,m} \left[ f_l r(l,m) - c(l,m)S(l,m)(t) \right] w(l,m)(t)   \\
\nonumber \text{ and }&  \\  \nonumber
 \alpha_2 (t) & =  \sum_{l \in \linkset,m}S(l,m)(t)w(l,m)(t)  \; \cdot   \\ 
&\Big[ c(l,m)- \expectation{ \big[C  (l,m)(t+1)\wedge x(l,m)(t)\big] \big| X(t) }  \Big]  
\end{align}
\end{small}
%These $\alpha_{\{1,2\}}$ are derived by first including in \eqref{expected_short} the $0$-valued term $[c(l,m)S(l,m)(t)w(l,m)(t) - c(l,m)S(l,m)(t)w(l,m)(t)]$, and then assuming that $S(l,m)(t) \in \{0,1\}$ (hence it can be brought outside of the internal min function in $\alpha_2$ without changing the ultimate result). %The later assumption does not hold in our modified controllers. 
\begin{Lem} \label{alpha2bound}
For all $l$, $m$, $t$, 
\begin{equation} 
\alpha_2 (t) \leq \sum_{l \in \linkset,m} c(l,m)\overline{C} (l,m)
\end{equation}
where $\overline{C} (l,m)$ is the maximum value of the random service rate $C(l,m)(t)$. 
\end{Lem}
%\underline{Proof of Lemma \ref{alpha2bound}:} \\
%By Jensen's inequality, 
%\begin{align*}
%\expectation{C(l,m)(t+1) & \wedge x(l,m)(t) \big| X(t)} \\
%& \leq \expectation{C(l,m)(t+1) \big| X(t)}  \wedge x(l,m)(t) \\
%&= c(l,m)\wedge x(l,m)(t) \\ 
%&\leq c(l,m)
%\end{align*}
% Furthermore, we know that the term 
%$ \left[ c(l,m)- \expectation{ \big[C(l,m)(t+1)\wedge x(l,m)(t)\big] \big| X(t) }  \right] $
%is non- negative, and only equal to $0$ when $x(l,m)(t)>\overline C (l,m)$. 
%Using these relations and the observations that $w(l,m)(t) \leq x(l,m)(t)$ and $S(l,m)(t)\in [0,1]$, the following must hold 
%\begin{align*}
%\alpha_2 (t) & =  \sum_{l \in \linkset,m}  S(l,m)(t)w(l,m)(t) \; \cdot  \\
% & \qquad  \left[ c(l,m)- \expectation{ \big[C(l,m)(t+1)\wedge x(l,m)(t)\big] \big| X(t) }  \right] \\
%& \leq \sum_{l \in \linkset,m} S(l,m)(t)x(l,m)(t) \; \cdot \\ 
%& \qquad \left[ c(l,m)- \expectation{ \big[C(l,m)(t+1)\wedge x(l,m)(t)\big] \big| X(t) }  \right] \\
%& \leq \sum_{l \in \linkset,m} c(l,m) \overline{C}(l,m)
% \end{align*}
The proof of Lemma \ref{alpha2bound} again follows \cite{MaxPressureStochastic} directly, as an extension from a binary controller $S$ to an $S\in [0,1]$ is trivial. In fact, the modifications made in the cycle-based max pressure controller only affect the $\alpha_1 (t)$ term, as we see in the proof of the following lemma. 

%%%%%%%%%%%%%%%%%%%%%%%%%%%%%%%%%%%%%%%%%%%%%%%%%%%%%%%%%%%%%%%%%
%%%%%%%%% %%%%%%%%%%%%%     BEGIN REAL PROOF HERE    %%%%%%%%%%%%%%%%%%%%%%%%%
%%%%%%%%%%%%%%%%%%%%%%%%%%%%%%%%%%%%%%%%%%%%%%%%%%%%%%%%%%%%%%%%%
 \begin{Lem} \label{alpha1bound}
If a relaxed max pressure control policy $u^{c*}$ is applied at time step $t$ and the demand $d$ is in the set of feasible undersaturated demands $D^\kappa$, then there exists an $\varepsilon>0$, $\eta>0$ such that 
\begin{equation} 
\alpha_1(t)  \leq -\varepsilon \eta \big| X(t)\big| 
\end{equation}
\end{Lem}
\underline{Proof of Lemma \ref{alpha1bound}:} \\
Consider the relaxed control matrix $S^{r*}$ specified by \eqref{distMP_LP}. By construction, it must be true that $ \forall S^r \in conv_{\kappa}$, 
\begin{align} \nonumber
\sum_{l,m}c(l,m)& w(l,m)(X(t))S^r (l,m) \\
&\leq \sum_{l,m}c(l,m)w(l,m)(X(t))S^{r*}(l,m) 
\end{align}
with equality only if $S^r = S^{r*}$. 
Therefore $\forall S^r \neq S^{r*}$,
\begin{align} \nonumber
 \sum_{l,m}&\big[f_{l}r(l,m) - c(l,m)S^{r*}(l,m)(t)\big]w(l,m)(X(t)) \\
&<   \sum_{l,m}\big[f_{l}r(l,m) - c(l,m)S^r(l,m)\big]w(l,m)(X(t))
\end{align}
If the demand flow is admissible according to \eqref{admissible_relaxed}, then 
$\exists \Sigma \in conv_{\kappa}$ such that 
\begin{equation} \nonumber
c(l,m)\Sigma(l,m) = \begin{cases}
        f_{l}r(l,m) + \varepsilon & \text{ if } w(l,m)(X(t)) > 0 \\
        0 & \text{ otherwise}
    \end{cases}
\end{equation}
{\color{red} Hence following the same logic as in \eqref{alpha1_w}, }
\begin{align} \nonumber
 \sum_{l,m}&\big[f_{l}r(l,m) -  c(l,m)S^{r*}(l,m)(t)\big] w(l,m)(X(t))  \\ \nonumber
&  < -\varepsilon \sum_{l \in \linkset,m} \max\{w(l,m)(X(t)) ,0\} \\
&\qquad +  \sum_{l \in \linkset,m}  f_l r(l,m)  \min\{ w(l,m)(X(t)) ,0\}
\end{align}
We assume that by our choice of $\sigma(l,m)$, $f_{l}r(l,m) > \varepsilon$ (omitting the cases where $r(l,m) = 0$). Therefore:
\begin{align}\nonumber
\sum_{l,m}\big[ f_{l}r(l,m) & - c(l,m)S^{r*}(l,m)(t)\big]w(l,m)(X(t)) \\
&< -\varepsilon \displaystyle\sum_{l,m} x(l,m)(t)
\end{align}

\subsection*{Explicit bound on queues, immediate feedback MP}
{\color{red} Combining \eqref{alphabound} and \eqref{betabound}, we obtain}
\begin{align}
\expectation{&|X(t+1)|^{2} - |X(t)|^{2}  |   X(t)} = \expectation{2\alpha(t) + \beta(t)} \nonumber \\
&<  -2\varepsilon \eta \left| X(t ) \right| + 2 \sum_{l \in \linkset,m} [c(l,m)\overline{C} (l,m) ]+ NB^{2}
\end{align}
where $N$ is the number of links in the network and $B=\max\left\{ \overline{C}(l,m), \sum_{k} \overline{C}(k,l),  \overline d (l,m) \right\}$. 
For simplicity, we combine all constant additive terms to define a new constant $K$:
\begin{align}\label{constCalc}
K = 2 \displaystyle\sum_{l,m} & c(l,m) \overline{C}(l,m) \\
\nonumber & + N \displaystyle\sum_{l,m} \max\left\{ \overline{C}(l,m), \sum_{k} \overline{C}(k,l),  \overline d (l,m) \right\}^2
\end{align}
So, when the control is updated at time $t$:
\begin{equation}\label{stabBasic}
\expectation{|X(t+1)|^{2} - |X(t)|^{2}  |   X(t)} < -2\varepsilon \eta \left| X(t ) \right| + K
\end{equation}
%Therefore we have an inequality of the form
%\begin{equation}\label{stabBasic}
%\expectation{|X(t+1)|^{2} - |X(t)|^{2}  |   X(t)} < -\varepsilon \eta \left| X(t ) \right| + K
%\end{equation}
%where 
%\begin{equation}\label{constCalc}
%K = 2\displaystyle\sum_{l,m} c(l,m) \overline{C}(l,m) + \displaystyle\sum_{l,m} \max[\overline{C}(l,m),\sum_{k}\overline{C}(k,l)]^2
%\end{equation}




%%%%%%%%%%%%%%%%%%%%%%%%%%%%%%%%%%%%%%%%%%%%%%%%%%%%%%%%%%%





%Consider the relaxed control matrix $S^{r*}$ specified by \eqref{distMP_LP}. By construction, it must be true that $ \forall S^r \in conv_{\kappa}$, 
%\begin{align} \nonumber
%\sum_{l,m}c(l,m)& w(l,m)(X(t))S^r (l,m) \\
%&\leq \sum_{l,m}c(l,m)w(l,m)(X(t))S^{r*}(l,m) 
%\end{align}
%with equality only if $S^r = S^{r*}$. 
%Therefore $\forall S^r \neq S^{r*}$,
%\begin{align} \nonumber
% \sum_{l,m}&\big[f_{l}r(l,m) - c(l,m)S^{r*}(l,m)(t)\big]w(l,m)(X(t)) \\
%&<   \sum_{l,m}\big[f_{l}r(l,m) - c(l,m)S^r(l,m)\big]w(l,m)(X(t))
%\end{align}
%If the demand flow is admissible according to \eqref{admissible_relaxed}, then 
%$\exists \Sigma \in conv_{\kappa}$ such that 
%\begin{equation} \nonumber
%c(l,m)\Sigma(l,m) = \begin{cases}
%        f_{l}r(l,m) + \varepsilon & \text{ if } w(l,m)(X(t)) > 0 \\
%        0 & \text{ otherwise}
%    \end{cases}
%\end{equation}
%Hence following the same logic as in \eqref{alpha1_w}, 
%\begin{align} \nonumber
% \sum_{l,m}&\big[f_{l}r(l,m) -  c(l,m)S^{r*}(l,m)(t)\big] w(l,m)(X(t))  \\ \nonumber
%&  < -\varepsilon \sum_{l \in \linkset,m} \max\{w(l,m)(X(t)) ,0\} \\
%&\qquad +  \sum_{l \in \linkset,m}  f_l r(l,m)  \min\{ w(l,m)(X(t)) ,0\}
%\end{align}
%We assume that by our choice of $\sigma(l,m)$, $f_{l}r(l,m) > \varepsilon$ (omitting the cases where $r(l,m) = 0$). Therefore:
%\begin{align}\nonumber
%\sum_{l,m}\big[ f_{l}r(l,m) & - c(l,m)S^{r*}(l,m)(t)\big]w(l,m)(X(t)) \\
%&< -\varepsilon \displaystyle\sum_{l,m} x(l,m)(t)
%\end{align}
%

We then establish a bound on the incremental queue differences within a cycle of length $\tau$, following the form of \eqref{stability_sufficient}: 
\begin{Lem} \label{lemma_p} 
For a given cycle (i.e. when the controller is not updated) consisting of time steps $\{t, t+1, \ldots,  t+\tau\}$,  $\forall p \in [0, \tau - 1 ]$, 
\begin{align} \nonumber 
&\expectation{\vert X(t + p + 1)\vert^2 - \vert X(t + p)\vert^2 \vert X(t) \ldots X(t + p - 1)} \\
&< -2\varepsilon \eta \vert X(t + p )\vert + B(p) + K \label{boundCycleMP}
\end{align}
where
\begin{align}  \label{Kdef}
K = 2 \sum_{l,m}&c(l,m)\overline{C}(l,m) \\ \nonumber 
& + N \sum_{l,m} \max\{ \overline{C}(l,m),\sum_{k}\overline{C}(k,l), \overline{d}(l,m) \} ^2
\end{align}
and
\begin{align} \nonumber 
B(p) &= p\left(2\varepsilon \eta  \sum_{l,m} + 2  (\sum_{l,m}[f_{l}r(l,m) + c(l,m)]) \right) \\
& \qquad \qquad \cdot \max\left\{ \overline{C}(l,m),\sum_{k}\overline{C}(k,l),\overline d(l,m) \right\} \label{Cpdef}
\end{align}
\end{Lem}
\underline{Proof of Lemma \ref{lemma_p}:} \\
As above, we have:
\begin{align} \label{immediate_to_bound} 
 |X(t+ p + 1)|^{2} &- |X(t + p)|^{2} \\  \nonumber
 & = 2(\alpha_1(t + p)+\alpha_2(t + p)) + \beta(t + p)   
\end{align}
where $\beta$, $\alpha_{1}$ and $\alpha_{2}$ are quantities that depend on the controller applied at time step $t + p$, as defined in Section \ref{sec:immediatefeedback}:
\begin{align*}
\beta (t+p)  &= \vert X(t + p + 1) - X(t + p) \vert^{2} \\
\alpha_{1} (t+p) &=  \sum_{l,m} \Big(f_{l}r(l,m) - c(l,m)S(l,m)(t)\Big)\\ 
& \qquad \qquad \cdot w(l,m)(X(t + p)) \\
 \alpha_{2}  (t+p)&= \sum_{l,m} \bigg(c(l,m)S(l,m)(t) \\ 
 - \expectation{ & \big[C(l,m)(t+p + 1)\wedge x(l,m)(t + p)\big] \big| X(t + p) }\bigg)\\
&\qquad \qquad \cdot w(l,m)(X(t + p))
 \end{align*}
As previously derived, the following bounds on $\beta(\cdot)$ and $\alpha_2(\cdot)$ will hold for any binary control matrix: 
\begin{align}
%\label{a1bound_original} \alpha_1 &<  -\varepsilon \eta \left| X(t ) \right| \\ 
\label{a2bound_original} \alpha_2 (\cdot)&<  \sum_{l \in \linkset,m} c(l,m)\overline{C} (l,m) \\ 
\label{bbound_original} \beta (\cdot)  &<  N \displaystyle\sum_{l,m} \max\left\{ \overline{C}(l,m), \sum_{k} \overline{C}(k,l),  \overline d (l,m) \right\}^2
\end{align}
These two terms form the constant $K$ from \eqref{constCalc}, which also appears in \eqref{boundCycleMP}. To complete the bound in  \eqref{boundCycleMP} we are only left with the $\alpha_1$ term, which is directly dependent on the explicit form of the binary controller $S$:

\begin{align}
\nonumber
&\expectation{\vert X(t + p + 1)\vert^2 - \vert X(t + p)\vert^2 \vert X(t) \ldots X(t + p)} \\
\nonumber
& = \expectation{ 2\alpha_{1} (t+p) + 2\alpha_{2}(t+p) +\beta (t+p)    \vert  X(t) \ldots X(t + p)  }\\
\nonumber
&< \expectation{2\alpha_{1} (t+p) \vert X(t) \ldots X(t + p) } + K \\
&= 2\sum_{l,m} {[f_{l}r(l,m) - c(l,m)S(l,m)(t) ]w(l,m)(X(t + p))} + K \label{withK}
\end{align}

%%Therefore we try to show that  $\forall p \in [0,\tau - 1 ]$, $\exists C(p)$ such that
%\begin{equation}
%\expectation{\alpha_{1}(t + p)\vert X(t)} < -\varepsilon \vert X(t) \vert  + C(p)
%\end{equation}
Examine the remaining  term, $\alpha_1(t)$: %$2\sum_{l,m} [f_{l}r(l,m) - c(l,m)S(l,m)(t) ]w(l,m)(X(t + p) )$:
\begin{align} \nonumber 
&2\sum_{l,m} [f_{l}  r(l,m) -  c(l,m)S(l,m)(t) ]  w(l,m)(X(t + p))   \\ \nonumber
&=  2\sum_{l,m} [f_{l}r(l,m) - c(l,m)S(l,m)(t) ]w(l,m)(X(t) )  \\ \nonumber
& \qquad  + 2\sum_{l,m} [f_{l}r(l,m) - c(l,m)S(l,m)(t) ] \\ \nonumber 
&\qquad \cdot \big(w(l,m)(X(t + p ) - X(t))\big) \\ \nonumber
&= 2 \xi_{1} + 2 \xi_{2}
\label{2terms}
\end{align}
With
\begin{align} 
\xi_{1} (t,S) = \sum_{l,m} \left[f_{l}r(l,m) - c(l,m)S(l,m)(t) \right]w(l,m)(X(t) )
\end{align}
and
\begin{align}\nonumber
\xi_{2} (t,p,S) &= \sum_{l,m} \left[f_{l}r(l,m) - c(l,m)S(l,m)(t) \right]\\
&\qquad \cdot\big(w(l,m)(X(t + p ) - X(t))\big)
\end{align}
\underline{Bound on $\xi_{1}$} \\
By Lemma \ref{alpha1bound} we know that
\begin{equation*}
2\sum_{l,m} [f_{l}r(l,m) - c(l,m)S(l,m)(t) ]w(l,m)(X(t )) < -2\varepsilon \eta \vert X(t) \vert
\end{equation*}
Then noting that
\begin{align*}
\vert X(t) \vert &= \vert X(t + p) - (X(t + p) - X(t)) \vert \\
&> \big\vert \vert X(t + p) \vert - \vert  X(t + p) - X(t)   \vert  \big\vert \\
&>  \vert X(t + p) \vert - \vert  X(t + p) - X(t)   \vert  
\end{align*}
we are left with
\begin{align} \nonumber
2\xi_1 (t,S) &< -2\varepsilon \eta \big( \vert X(t + p) \vert -  \vert  X(t + p) - X(t)   \vert \big)\\ \nonumber
&< -2\varepsilon \eta   \vert X(t + p) \vert \\  \nonumber
&\qquad + 2\varepsilon \eta\sum_{i=1}^{p} \vert  X(t + i) - X(t + i - 1)   \vert  \\ 
&= -2\varepsilon \eta   \vert X(t + p) \vert + 2\varepsilon \eta\sum_{i=1}^{p} \vert \delta(t + i -1) \vert 
\label{sump}
\end{align}
So by \eqref{sump} and \eqref{betabound}, 
\begin{align}  \label{first_term}
2\sum_{l,m} & [f_{l}r(l,m) - c(l,m)S(l,m)(t) ]w(l,m)(X(t))  \\ \nonumber
&  < -2\varepsilon \eta  \vert X(t + p) \vert \\ \nonumber
& \qquad + 2\varepsilon \eta  p \sum_{l,m} \max\left\{ \overline{C}(l,m),\sum_{k}\overline{C}(k,l),\overline d(l,m) \right\} 
\end{align}
Plugging \eqref{first_term} into \eqref{withK}, we have

\begin{align} \nonumber
&\expectation{\vert X(t + p +  1)\vert^2 - \vert  X(t + p)\vert^2 \vert X(t),\ldots,X(t + p)}  \\ \nonumber
& < K  -2 \varepsilon \eta \vert X(t + p) \vert  \\ \nonumber
&\qquad + 2\varepsilon \eta p \sum_{l,m} \max\left\{ \overline{C}(l,m),\sum_{k}\overline{C}(k,l),\overline d(l,m) \right\} \\ \nonumber
&  \qquad +2 \sum_{l,m} [f_{l}r(l,m) - c(l,m)S(l,m)(t) ]\\
&\qquad \qquad \cdot\Big(w(l,m)(X(t + p) ) - w(l,m)(X(t))\Big) \label{combined1}
\end{align}
%
%\begin{align} \nonumber
%\sum_{l,m} [f_{l}  r(l,m) - & c(l,m)S(l,m)(t) ]  w(l,m)(t + p)   \\ \nonumber
%& =   \sum_{l,m} [f_{l}r(l,m) - c(l,m)S(l,m)(t) ]w(l,m)(t )  \\
%&\qquad \qquad \qquad +\sum_{l,m} [f_{l}r(l,m) - c(l,m)S(l,m)(t) ](w(l,m)(t + p ) - w(l,m)(t)) 
%\label{2terms}
%\end{align}
\underline{Bound on $\xi_{2}$} \\
We now have to bound the term
\begin{align}\nonumber
2\xi_2(t,p,S) &= 2\sum_{l,m} [f_{l}r(l,m) - c(l,m)S(l,m)(t) ]\\
&\quad \cdot \Big(w(l,m)X((t + p) ) - w(l,m)(X(t))\Big)
\label{2xi2}
\end{align}
%
%\begin{comment}
%
%\begin{align*}
%\expectation{ \vert X(t+p)\vert^2 - \vert X(t+ p - 1)\vert^2  \vert  X(t) } &= \sum_{l,m}[f_{l}r(l,m) - c(l,m)S(l,m)]w(l,m)(X(t + p)) + \frac{K}{p^2}\\
%&=   \sum_{l,m}[f_{l}r(l,m) - c(l,m)S(l,m)]  w(l,m)(X(t + k)) + \frac{K}{p^2}\\
%%&= \tau K + \sum_{l,m}[f_{l}r(l,m) - c(l,m)S(l,m)]\sum_{k=1}^{\tau - 1}  
%%\{ w(l,m)(X(t) ) + w(l,m)(X(t + k )) - w(l,m)(X(t) ) \} \\
%&=  K + \sum_{l,m}[f_{l}r(l,m) - c(l,m)S(l,m)]w(l,m)(X(t)) \\
%&+ \sum_{l,m}[f_{l}r(l,m) - c(l,m)S(l,m)]  
%\{ w(l,m)(X(t + k) ) - w(l,m)(X(t) ) \} \\
%&< K - \varepsilon \vert X(t) \vert + \sum_{l,m}[f_{l}r(l,m) - c(l,m)S(l,m)]  
%\{ w(l,m)(X(t + k) ) - w(l,m)(X(t) ) \} \\
%\end{align*}
%
%\end{comment}
For that purpose we study the term
\begin{align} \nonumber
&w(l,m)(X(t+p))   -  w(l,m)(X(t)) \\ \nonumber
& = \sum_{n=1}^{p} w(l,m)(X(t+n)) - w(l,m)(X(t + n - 1)) \\ \nonumber
&= \sum_{n=1}^{p} \Big\{ x(l,m)(t + n) - x(l,m)(t + n - 1)   \\ \nonumber 
&  \quad - \sum_{s \in Out(m)}[ x(m,s)(t + n) - x(m,s)(t + n - 1) ]r(m,s)  \Big\} \\ 
&= \sum_{n=1}^{p}  w(l,m)( \delta(t + n - 1)) 
\end{align}
{\color{red}
By \eqref{deltabound2} and the fact that $w(\cdot)$ is linear,  }
\begin{align} \nonumber
\vert w(l,m)(\delta&(t + n -1)) \vert \\  \label{wdelta}
&< \sum_{u,v} \max\Big\{  \overline{C}(u,v), \sum_{k}\overline{C}(k,u), \overline{d}(u,v)\Big\}
\end{align}
Therefore plugging \eqref{wdelta} back into \eqref{2xi2}, we get
\begin{align} \nonumber 
2\xi_2&(t,p,S) = 2\bigg( \sum_{l,m}\big([f_{l}r(l,m) - c(l,m)S(l,m)] \\  \nonumber
&\qquad \qquad \qquad \cdot \sum_{n=1}^{p}  w(l,m)( \delta(t + n - 1))\bigg)  \\ \nonumber
&< 2 \sum_{n=1}^{p}  \sum_{l,m}[f_{l}r(l,m) - c(l,m)S(l,m)] \\ 
&\qquad \qquad \cdot\sum_{u,v} \max\Big\{  \overline{C}(u,v), \sum_{k}\overline{C}(k,u), \overline{d}(u,v)\Big\} \label{2xi2_bound1}
\end{align}
Also note that 
\begin{equation}
\vert \sum_{n=1}^{p}  \sum_{l,m}[f_{l}r(l,m) - c(l,m)S(l,m)] \vert < p \sum_{l,m}[f_{l}r(l,m) + c(l,m)]
\end{equation}
so \eqref{2xi2_bound1} becomes
\begin{align} \label{2xi2_bound2}
2\xi_2(t,p,S) < 2 p &\left( \sum_{l,m}[f_{l}r(l,m) + c(l,m)]\right)\\ \nonumber
&\cdot\left( \sum_{l,m} \max\Big\{  \overline{C}(l,m), \sum_{k}\overline{C}(k,l), \overline{d}(l,m)\Big\} \right)
\end{align}

Substituting \eqref{2xi2_bound2} into \eqref{combined1} yields the final bound expressed in
 \eqref{boundCycleMP}:

\begin{align} \nonumber
\expectation{&\vert X(t + p +  1)\vert^2 - \vert  X(t + p)\vert^2 \vert X(t),\ldots,X(t + p)}  \\ \nonumber
&<  K  -2 \varepsilon \eta \vert X(t + p) \vert \\ \nonumber
&\qquad + 2\varepsilon \eta p \sum_{l,m} \max\left\{ \overline{C}(l,m),\sum_{k}\overline{C}(k,l),\overline d(l,m) \right\}  \\ \nonumber
&  \qquad  + 2 p \left(\sum_{l,m}[f_{l}r(l,m) + c(l,m)]\right) \\ \nonumber
& \qquad \qquad \cdot \left( \sum_{l,m} \max\Big\{  \overline{C}(l,m), \sum_{k}\overline{C}(k,l), \overline{d}(l,m)\Big\} \right) \\
&= K  -2 \varepsilon \eta \vert X(t + p) \vert + B(p)
\end{align}
with $K$ and $B(p)$ given by \eqref{Kdef} and \eqref{Cpdef}, respectively. 



Once we establish Lemma \ref{lemma_p}, we can show that for a time step $t$ within any number of cycles $T$, the following quantity is bounded:
\begin{align} \nonumber 
&\sum_{t = 1}^{\tau T} \expectation{\vert X(t + 1)\vert^2 - \vert X(t)\vert^2 \vert X(t)}\\ \nonumber
&=\sum_{t=1}^{T-1} \sum_{p=0 }^{\tau - 1} \expectation{\vert X(t + p + 1)\vert^2 - \vert X(t + p)\vert^2 \vert X(t + p)}\\ \nonumber
&< \sum_{t=1}^{T-1} \sum_{p=0 }^{\tau - 1} (-2\varepsilon\eta \vert X(t + p )\vert + B(p) + K ) \\
&< -2\varepsilon \eta \sum_{t=1}^{\tau T} \vert X(t )\vert  + (T - 1) \left(\sum_{p=0 }^{\tau - 1} B(p) + \tau K\right) \label{equivalencestabcycleMP}
\end{align}
Which, when taking the expectation, yields
\begin{align} \nonumber 
& \expectation{\vert X(\tau T + 1)\vert^2 - \vert X(1)\vert^2 } \\
& <  -2\varepsilon\eta \sum_{t=1}^{\tau T} \expectation{\vert X(t )\vert}+ (T - 1) ( \sum_{p=0 }^{\tau - 1} B(p) + \tau K)
\end{align}
Or, rearranging we obtain
\begin{align} \nonumber 
\dfrac{1}{\tau T}& \sum_{t=1}^{\tau T}  \expectation{\vert X(t )\vert} < \dfrac{1}{2\varepsilon\eta\tau T}\expectation{\vert X(1)\vert^2 - \vert X(\tau T + 1)\vert^2  }\\ \nonumber 
&\qquad \qquad \qquad \qquad \qquad + \dfrac{ T -1}{2\varepsilon\eta\tau T} \left(\sum_{p=0 }^{\tau - 1} B(p) + \tau K\right) \\
&< \dfrac{1}{2\varepsilon\eta\tau T}\expectation{\vert X( 1)\vert^2}
+ \dfrac{ 1}{2\varepsilon\eta\tau } \left(\sum_{p=0 }^{\tau - 1} B(p) + \tau K\right)
\end{align}
This establishes stability: the quantity $\dfrac{1}{\tau T}\sum_{t=1}^{\tau T} \expectation{\vert X(t )\vert}$ must then be bounded.\\ 

\begin{align} \label{CYCLE_BOUNDS}
\dfrac{1}{\tau T} \displaystyle\sum_{t=1}^{\tau T} \expectation{\vert X(t)}< \dfrac{1}{2\varepsilon\eta\tau T}& \expectation{\vert X(1) \vert^{2}} \\ \nonumber
& + \dfrac{1}{2\varepsilon\eta\tau}\displaystyle\sum_{p=0}^{\tau - 1} (B(p) + \tau K)
\end{align}

where $\varepsilon$, $\eta$ and $K$ are constants which depend on network topology and demand.

\end{proof}

The proof above shows that the system is stable.


%
%
%\subsection*{Stability of allocated max pressure}
%
%We now show that the allocated max pressure controller $S^{r*}$ is stable in the sense of \eqref{stability_sufficient} under conditions slightly modified from those assumed in the stability proofs of immediate feedback and cycle max pressure.
%


%%%%%%%%%%%%%%%%%%%%%%%%
%cycle allocated MP
%%%%%%%%%%%%%%%%%%%%%%%%


%\subsection*{Definition}
%
%We now combine the cycle max pressure and allocated max pressure: once every $\tau$ model time steps, a relaxed controller $S^r$ is chosen and then applied at each of the subsequent $\tau$ time steps. In other words, a single relaxed controller $S^r(t)$ is applied for the set of time steps $\{t+1, t+2, \ldots, t+\tau -1 \}$. Then, at time $t+\tau$, a new relaxed controller $S^r (\tau)$ is calculated to be applied at $\{t+\tau+1, \ldots, t+2\tau -1 \}$. 
%
%\subsection*{Stability of cycle-allocated max pressure}
%\begin{Thm}\label{StabCycleDistributedMP}
%The sequence of cycle-allocated max pressure controller stabilizes the network
% whenever the average demand vector $d = \lbrace d_{l}\rbrace$ is within the set of feasible demands
%$ D_{\kappa}$ defined above.
%%Under this condition, the quantity:
%%\begin{equation}
%%\dfrac{1}{T}\sum_{t=1}^{T}\expectation{\vert X(t) \vert}
%%\end{equation}
%%is bounded for any time horizon $T$.
%\end{Thm}
%
%\vspace{0.5cm}
%
%The stability of the cycle-allocated max pressure control follows from a logical combination of the stability proofs of cycle and allocated variations. Explicitly, the result for cycle max pressure in Appendix \ref{cycleproof} still holds in the case of a relaxed controller when one assumes that demand flow satisfies the modified cycle-by-cycle admissibility condition introduced for allocated max pressure in Section \ref{sec:distributed}. 



\subsection*{Increased bound on queue length}
Max pressure controllers that are only updated every $\tau$ model time steps will stabilize a network; however the resulting bound on the queues will be higher than in the immediate feedback setting. 
Comparing \eqref{IF_BOUNDS} and \eqref{CYCLE_BOUNDS}, note that the constants are increased by a factor of
\begin{equation}
\dfrac{1}{\varepsilon \tau} \sum_{p=1}^{\tau - 1} B(p) = \dfrac{1}{\varepsilon \tau}  \dfrac{\tau(\tau - 1)}{2}\cdot(\text{constant}) = \dfrac{\tau - 1}{2\varepsilon} \cdot(\text{constant})
\end{equation}
The relative cost to queue bounds is therefore linear in $\tau$. 


%%%%%%%%%%%%%%%%%%%%%%%%%%%%
%%%%%%%%%%%%%%%%%%%%%%%%%%%%










