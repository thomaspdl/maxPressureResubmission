% !TEX root = ./MParticle.tex
\label{app_oldproof}

Consider the expectation of the following function of queue state with perturbation 
\begin{equation} \label{delta_def}
\delta(t) = X(t+1) - X(t)
\end{equation}  
conditioned on the past queue state:
\begin{align}
\vert X(t+1) \vert^2  - \vert X(t)\vert^2 &= \vert X(t) + \delta (t) \vert^2 - \vert X(t)\vert^2 \\
\nonumber &= 2X(t)^{T}\delta(t) + \vert \delta(t)  \vert^2 \\
\nonumber &= 2\alpha(t) + \beta(t) 
\end{align}
with 
\begin{equation} \label{alpha_beta_defs}
\alpha(t)  = X(t)^{T}\delta(t)  \text{   \; and \;  }\beta(t) = \vert \delta (t) \vert^{2}
\end{equation}
We continue by addressing bounds on $\beta$ and $\alpha$ separately. 

\subsection*{Bound on $\beta(t) = \vert \delta (t)\vert^{2}$}
Define $\overline{C}$ as the maximum realized saturation flow and $\overline{d}$ as the maximum possible value of the demand vector. 
If $l \in \entrylinks$ and $m \in Out(l)$,
\begin{align} \nonumber
\big| \delta(l,m)(t)  \big| &= \bigg| -[C(l,m)(t+1)S(l,m)(t) \wedge x(l,m)(t)]  \\ \nonumber
&\qquad \qquad + D(l,m)(t+1) \bigg| \\
& \leq \max\left\{  \overline{C}(l,m) , \overline d (l,m) \right\} \label{deltabound1}
\end{align}
where $D(l,m)(t)  = D(l) (t) R(l,m)(t)$ with $D(l)(t)$ defined as the realized demand on link $l$ at time $t$. 
This is because we know that both $C(l,m)(t+1)S(l,m)(t) \wedge x(l,m)(t)$ and $D(l,m)(t+1)$ are non-negative, so the absolute value of their difference must be less than either of the two quantities individually. 

Similarly, if $l \in \linkset\backslash\entrylinks$ and $m \in Out(l)$:
\begin{align} \nonumber  
\big | \delta(l,&m) (t)  \big |  = \\ \label{deltabound}
 & \bigg\vert-\big[C(l,m)(t+1)S(l,m)(t) \wedge x(l,m)(t)\big]  \\  \nonumber
 & + \sum_{k}\big[C(k,l)(t+1)S(k,l)(t) \wedge x(k,l)(t)\big] R(l,m)(t+1)\bigg\vert  \\
& \leq \max\left\{  \overline{C}(l,m) , \sum_{k} \overline{C}(k,l) \right\} \label{deltabound2}
\end{align}

If we define $B$ as the maximum of all of the quantities $\left\{ \overline{C}(l,m), \sum_{k} \overline{C}(k,l),  \overline d (l,m) \right\}$ and $N$ as the number of queues in the network, we can derive a bound for $\beta$ which depends on only $B$ and $N$: 
\begin{equation}\label{betabound}
\beta(t)  = \big| \delta(t) \big|^2 \leq NB^2 
\end{equation}

Note that because these bounds hold for any $S(l,m)(t) \in [0,1]$, the bound on $\beta$ presented here can easily be extended to 
any convex combination of control matrices; hence it is still valid in our modified controllers, as shown later in this article.
%Such an extension is not as trivial with the $\alpha$ term. 

\subsection*{Bound on $\alpha (t) = X(t)^{T}\delta(t) $}
The term $\alpha$ in \eqref{alpha_beta_defs} is explicitly defined in terms of queue state $X(t)$ as follows: 

\begin{align} \nonumber 
 \alpha (t)  &=  X(t)^{T}\left[X(t+1) - X(t)\right] \\ \nonumber
 &=  \sum_{l \in \linkset \backslash \entrylinks ,m}\sum_{k}\left[ C(k,l)(t+1)S(k,l)(t)\wedge x(k,l)(t)\right]\\ \nonumber
&\qquad \qquad  \quad \cdot R(l,m)(t+1)x(l,m)(t) \\ \nonumber
&  \qquad \qquad -  \sum_{l \in \linkset,m}\left[ C(l,m)(t+1)S(l,m)(t)\wedge x(l,m)(t)\right] \\ \nonumber
&\qquad \qquad-\sum_{l \in \entrylinks ,m} d(l,m)(t+1)x(l,m)(t)\\ \nonumber
&= \sum_{l \in \linkset,m}\big[C(l,m)(t+1)S(l,m)(t)\wedge x(l,m)(t)\big] \\ \nonumber 
& \qquad \qquad \cdot \Big(-x(l,m)(t)+  \sum_{p}R(m,p)(t+1)x(m,p)(t)\Big) \\  
 & \qquad \qquad + \sum_{l \in \entrylinks,m} d(l,m)(t+1)x(l,m)(t)  
 \label{alpha_explicit}
\end{align}
Note that only the expectation of these terms appear in equation \eqref{stability_sufficient}, so we are interested in $\mathbb{E}\{\alpha(t) | X(t)\}$. We therefore make the following observation: because $R(m,p)(t+1)$ is independent of $C(l,m)(t+1)$ and $X(t)$, 
\begin{align*}
\nonumber \expectation{&\big[C(l,m)(t+1)S(l,m)(t)\wedge x(l,m)(t)\big] \\
\nonumber & \qquad \quad  \;   \cdot R(m,p)(t+1)x(m,p)(t) \big| X(t)}  \\
\nonumber & = \expectation{\big[C(l,m)(t+1)S(l,m)(t)\wedge x(l,m)(t)\big]  \big| X(t)} \\
\nonumber & \qquad \quad  \; \cdot r(m,p)(t+1)x(m,p)(t)
\end{align*}

\vspace{0.5cm}

Also, the expectation of demand $d(l,m)$ is equal to the measured demand $d_{l}$ on link $l$, times the relevant expected split ratio $r(l,m)$. 
Hence the desired expectation of \eqref{alpha_explicit} can be expressed as
\begin{align} 
\nonumber \mathbb{E} & \{\alpha (t) | X(t)\}= \\
\nonumber & \sum_{l \in \linkset,m} \expectation{ \big[C(l,m)(t+1)S(l,m)(t)\wedge x(l,m)(t)\big]\big| X(t)}\\
\nonumber & \qquad \quad  \;   \cdot \Big(-x(l,m)(t) + \sum_{p}r(m,p)(t+1)x(m,p)(t)\Big) \\
 \nonumber &\qquad + \sum_{l \in \entrylinks,m} d_l r(l,m) x(l,m)(t)  \\
 \nonumber =& \sum_{l \in \entrylinks,m} d_l r(l,m) x(l,m)(t) \\
\nonumber &- \sum_{l \in \linkset,m} \expectation{ \big[C(l,m)(t+1)S(l,m)(t)\wedge x(l,m)(t)\big] \big| X(t)} \\  
\label{Ea} &\qquad \qquad \cdot w(l,m)(t)
\end{align}
where $w(l,m)(t) = w(l,m)(X(t))$ is the weight of a link as defined in \eqref{linkweight}. 
We also can include the following relation: 
\begin{align*}
\sum_{l \in \linkset,m} f_l &r(l,m) w(l,m)(t) \\ 
& = \sum_{l \in \linkset,m} f_l r(l,m) \left[ x(l,m) - \sum_p r(m,p)x(m,p)(t) \right] \\
& = \sum_{l \in \linkset,m} f_l r(l,m)x(l,m)(t) \\
&\qquad \qquad - \sum_m  \left[ \sum_{l\in\linkset} f_l r(l,m) \sum_p r(m,p)x(m,p)(t) \right] \\
& = \sum_{l \in \linkset,m} f_l r(l,m)x(l,m)(t)\\
&\qquad \qquad  - \sum_{m\in \linkset \backslash \entrylinks, p}   f_m r(m,p)x(m,p)(t) \\ 
& =  \sum_{l \in \entrylinks,m} d_l r(l,m)x(l,m)(t)
\end{align*}
So \eqref{Ea} is further simplified to: 
\begin{align} \label{expected_short}
 & \mathbb{E}\{\alpha (t) | X(t)\} = \sum_{l \in \linkset,m} \bigg[ f_l r(l,m)  \\
 \nonumber  & - \expectation{ \big[C(l,m)(t+1)S(l,m)(t)\wedge x(l,m)(t)\big] \big| X(t)}  \bigg] w(l,m)(t)  \\  
 & \qquad \qquad = \alpha_1 (t) + \alpha_2 (t) \nonumber
\end{align}
with 
\begin{align} \label{a1_def}
\alpha_1 (t) &=  \sum_{l \in \linkset,m} \left[ f_l r(l,m) - c(l,m)S(l,m)(t) \right] w(l,m)(t)   \\
\nonumber \text{ and }&  \\ 
 \alpha_2 (t) & =  \sum_{l \in \linkset,m} \Big[ c(l,m)- \\ \nonumber
\expectation{ \big[C & (l,m)(t+1)\wedge x(l,m)(t)\big] \big| X(t) }  \Big] S(l,m)(t)w(l,m)(t)  
\end{align}
These $\alpha_{\{1,2\}}$ are derived by first including in \eqref{expected_short} the $0$-valued term $[c(l,m)S(l,m)(t)w(l,m)(t) - c(l,m)S(l,m)(t)w(l,m)(t)]$, and then assuming that $S(l,m)(t) \in \{0,1\}$ (hence it can be brought outside of the internal min function in $\alpha_2$ without changing the ultimate result). %The later assumption does not hold in our modified controllers. 
\begin{Lem} \label{alpha2bound}
For all $l$, $m$, $t$, 
\begin{equation} 
\alpha_2 (t) \leq \sum_{l \in \linkset,m} c(l,m)\overline{C} (l,m)
\end{equation}
where $\overline{C} (l,m)$ is the maximum value of the random service rate $C(l,m)(t)$. 
\end{Lem}
\underline{Proof of Lemma \ref{alpha2bound}:} \\
By Jensen's inequality, 
\begin{align*}
\expectation{C(l,m)(t+1) & \wedge x(l,m)(t) \big| X(t)} \\
& \leq \expectation{C(l,m)(t+1) \big| X(t)}  \wedge x(l,m)(t) \\
&= c(l,m)\wedge x(l,m)(t) \\ 
&\leq c(l,m)
\end{align*}
 Furthermore, we know that the term 
$ \left[ c(l,m)- \expectation{ \big[C(l,m)(t+1)\wedge x(l,m)(t)\big] \big| X(t) }  \right] $
is non-negative, and only equal to $0$ when $x(l,m)(t)>\overline C (l,m)$. 
Using these relations and the observations that $w(l,m)(t) \leq x(l,m)(t)$ and $S(l,m)(t)\in \{0,1\}$, the following must hold 
\begin{align*}
\alpha_2 (t) & = \sum_{l \in \linkset,m} \left[ c(l,m)- \expectation{ \big[C(l,m)(t+1)\wedge x(l,m)(t)\big] \big| X(t) }  \right] \\ &\qquad \qquad \qquad \cdot S(l,m)(t)w(l,m)(t) \\
& \leq \sum_{l \in \linkset,m} \left[ c(l,m)- \expectation{ \big[C(l,m)(t+1)\wedge x(l,m)(t)\big] \big| X(t) }  \right] \\ &\qquad \qquad \qquad \cdot S(l,m)(t)x(l,m)(t)   \\
& \leq \sum_{l \in \linkset,m} c(l,m) \overline{C}(l,m)
 \end{align*}



 \begin{Lem} \label{alpha1bound}
If the immediate feedback max pressure control policy $u^*$ is applied and the demand $d$ is in the set of feasible demands $D^o$, then there exists an $\varepsilon>0$, $\eta>0$ such that 
\begin{equation} 
\alpha_1(t)  \leq -\varepsilon \eta \big| X(t)\big| 
\end{equation}
\end{Lem}
\underline{Proof of Lemma \ref{alpha1bound}:} \\
Applying the definition of immediate feedback max pressure control in \eqref{original_MP} as $S^*$, 
\begin{align*}
\sum_{l,m} S^* (l,m)&(t) c(l,m) w(l,m)(t) \\
&= \max_{S\in U} \sum_{l,m} S(l,m)c(l,m)w(l,m)(t) \\
&= \max_{M\in co(U)} \sum_{l,m} M(l,m)c(l,m)w(l,m)(t)
\end{align*}
As in \eqref{feasible_demand}, since $d\in D^0$, there exists an $\varepsilon > 0$ and $M^+$ such that $C(l,m)M^+(l,m) > f_l r(l,m) + \varepsilon \; \forall (l,m)$. Logically, any $M'$ such that $0\leq M' \leq M^+$ (component-wise) must also be in $co(U)$. Therefore we choose a $M' < M^+$ such that 
\begin{equation} \label{epsDef}
M'(l,m) c(l,m) = \begin{cases} 
f_l r(l,m) + \varepsilon  & \text{ if } w(l,m)>0 \\
0 & \text{ if }w(l,m)\leq 0
\end{cases}
\end{equation} 
 Then
 \begin{align} %\label{alpha1_intermediate}
\nonumber  \alpha_1 (t) & =  \sum_{l \in \linkset,m} \left[ f_l r(l,m) - c(l,m)S^*(l,m)(t) \right] w(l,m)(t) \\
 \nonumber & \leq \sum_{l \in \linkset,m} \left[ f_l r(l,m) - M'(l,m)c(l,m) \right] w(l,m)(t) \\
\nonumber & = -\varepsilon \sum_{l \in \linkset,m} \max\{w(l,m)(t),0\} \\
\nonumber & \qquad \qquad +  \sum_{l \in \linkset,m}  f_l r(l,m)  \min\{ w(l,m)(t),0\} \\
& \leq -\varepsilon \sum_{l \in \linkset,m}  \left| w(l,m) (t) \right| \label{alpha1_w}
\end{align}
Notice that $w(l,m)(X(t))$ is a linear, invertible function of the array $X(t)$, and therefore there exists a $\eta >0$ such that $\sum_{l,m} | w(l,m) (X(t)) | \geq \eta |X(t)|$. Substituting this expression into \eqref{alpha1_w} defines a bound on $\alpha_1$ (t): 
\begin{equation} \label{a1_bound_eq}
\alpha_1(t) \leq -\varepsilon \eta \left| X(t ) \right|
\end{equation}
Combining the results of Lemmas \ref{alpha2bound} and \ref{alpha1bound} generates the desired bound on $\expectation{\alpha (t) | X(t)}$:
\begin{equation}\label{alphabound}
\expectation{\alpha(t) | X(t)} \leq -\varepsilon \eta \left| X(t ) \right| + \sum_{l \in \linkset,m} c(l,m)\overline{C} (l,m)
\end{equation}

\subsection*{Explicit bound on queues, immediate feedback MP}

Combining \eqref{alphabound} and \eqref{betabound}, we obtain
\begin{align}
\expectation{&|X(t+1)|^{2} - |X(t)|^{2}  |   X(t)} = \expectation{2\alpha(t) + \beta(t)} \nonumber \\
&<  -2\varepsilon \eta \left| X(t ) \right| + 2 \sum_{l \in \linkset,m} [c(l,m)\overline{C} (l,m) ]+ NB^{2}
\end{align}
where $N$ is the number of links in the network and $B=\max\left\{ \overline{C}(l,m), \sum_{k} \overline{C}(k,l),  \overline d (l,m) \right\}$. 
For simplicity, we combine all constant additive terms to define a new constant $K$:
\begin{align}\label{constCalc}
K = 2 \displaystyle\sum_{l,m} & c(l,m) \overline{C}(l,m) \\
\nonumber & + N \displaystyle\sum_{l,m} \max\left\{ \overline{C}(l,m), \sum_{k} \overline{C}(k,l),  \overline d (l,m) \right\}^2
\end{align}
So
\begin{equation}\label{stabBasic}
\expectation{|X(t+1)|^{2} - |X(t)|^{2}  |   X(t)} < -2\varepsilon \eta \left| X(t ) \right| + K
\end{equation}
%Therefore we have an inequality of the form
%\begin{equation}\label{stabBasic}
%\expectation{|X(t+1)|^{2} - |X(t)|^{2}  |   X(t)} < -\varepsilon \eta \left| X(t ) \right| + K
%\end{equation}
%where 
%\begin{equation}\label{constCalc}
%K = 2\displaystyle\sum_{l,m} c(l,m) \overline{C}(l,m) + \displaystyle\sum_{l,m} \max[\overline{C}(l,m),\sum_{k}\overline{C}(k,l)]^2
%\end{equation}
This is sufficient to show stability as previously defined.

%%%%%%%%
%END: Original proof
%%%%%%%%

%\subsection*{Comments for the extensions of the proof}
%
%To sum things up
%
%\begin{align*}
%\mathbb{E}[\vert X(t + 1) \vert - \vert X(t) \vert  \vert X(t)] &= 2\alpha(t) + \beta(t)\\
%&= 2\alpha_{1}(t) + 2\alpha_{2}(t) + \beta(t)\\
%&< 2\alpha_{1} + K
%\end{align*}
%
%since $\alpha_{2}(t)$ and $\beta(t)$ are bounded for any binary or relaxed controller.
%
%Therefore, if we manage to show a property such as 
%
%\begin{equation}
%\sum_{l,m} (f_{l}r(l,m) - c(l,m)S(l,m))w(l,m) < - \varepsilon \vert  X(t) \vert 
%\end{equation}
%
%for the controller, then stability instantanely follows.

